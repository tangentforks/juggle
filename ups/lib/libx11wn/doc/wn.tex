% @(#)wn.tex	1.1 4/11/91 (UKC)
% This document was written by John Bovey, University of Kent
%
\documentstyle[11pt]{article}
\oddsidemargin 0in
\evensidemargin 0in
\topmargin 0in
\textwidth 6.25in
\textheight 8.75in
\parindent 0in
\parskip 0.1in
\title{Using the wn graphics library}
\author{J D Bovey \and M T Russell}
\pagestyle{headings}
\newcommand{\funclabel}[1]{\label{#1}}
\begin{document}
\maketitle
\tableofcontents
\newpage
%%%%%%%%%%%%%%%%%%%%%%%%%%%%%%%%%%%%%%%%
\section{Introduction}
{\em Wn} is a library of low level graphical input and output routines
which is principally intended for writing graphical interactive applications
programs. There are versions of {\em wn} which run under X11 \cite{xlib}, X10
and Suntools (see \cite{sunviewprog,sunviewsysprog,pixrect}).
Although {\em wn} does not provide menus or any other high-level
graphical input device, it is compatible with the {\em med3} menu package \cite{med3doc}.
The remainder of this section is a general introduction
to the sort of graphical input and output supported by {\em wn}
and can probably be skipped by a reader who is familiar with workstation
programming.
The later sections describe the {\em wn} functions in detail.
%
%%%%%%%%%%%%%%%%%%%%%%%%%%%%%%%%%%%%%%%%
\subsection{Windows}
All input and output handled by {\em wn} is assumed to go via a window
on the screen. A program using {\em wn} does not need to know where the
window is or whether it is partially obscured by other windows
(although it does often need to know how large the window is); the position
of the window on the screen and its relationship to other windows 
will be handled by the window manager.
As far as the program using {\em wn} is concerned it has a rectangular
area of screen in which it can draw, and through which it can detect input
events like cursor movements and mouse button presses.
Under some window systems the program will be asked occasionally to
redraw the entire contents of its window.
This happens when parts of the window which had been obscured by
other windows are exposed.
%
%%%%%%%%%%%%%%%%%%%%%%%%%%%%%%%%%%%%%%%%
\subsection{Graphical output}
A window used for output can be considered as
a rectangular array of pixels with each pixel having a pair of
non-negative coordinates ({\em x},{\em y}).
Pixel (0,0) is in the top left hand corner of the window and {\em x} runs
from 0 to one less than the window width, {\em y} from 0 to one less than
the window height. Any graphical output addressed to pixels outside the
window will be {\em clipped} at the window boundary.

The number of {\em planes} used by the display is the number of bits used
to store a pixel.
A monochrome display will have one bit per pixel, whereas a color display
will typically have 8 bits per pixel giving a total of 256 possible pixel
values.
A pixel value does not represent a color directly but is used as an index
into a table of actual colors called the {\em colormap}.
An actual color is represented by a
triple of intensities, one for each of the three primary colors red, green and blue.
A color intensity normally runs from 0 to 255 and so there is a palette of $256^{3}$
or nearly 17 million different colors available (although there can be at
most 256 different colors on the screen at one time).
Since the same pixel value represents the same color, wherever it is on the
screen, and since there may be several programs simultaneously trying
to produce colored images in different windows,
the 256 different pixel values have to be thought of as a resource.
A program which wants to display {\em n} different colors will
request {\em n} pixel values from the window system and if successful will
set the colormap entries for the allocated pixel values to the desired colors.
The request will fail if there are currently less than {\em n} unused pixel values.
%%%%%%%%%%%%%%%%%%%%%%%%%%%%%%%%%%%%%%%%
\subsection{Graphical output primitives}
There are four major output primitives provided by functions in the
{\em wn} library.
More complex output operations can be built using these primitives
which are:
\begin{description}
\item[Draw a line.]
The application specifies the coordinates of the end points and the pixel
value of the line.
\item[Display a piece of text.]
The application specifies the location within the window, the string to be
displayed, the font used to display it and the pixel values.
The font is a data structure describing how each ASCII character will be drawn
on the screen.
In addition to the font height and character widths the font contains
an array of rectangular binary images with
one image for each ASCII character value.
Font character images are binary with the bit values describing which pixels
are drawn as foreground and which are drawn as background.
This means that in general the text drawing function needs to be given
two pixel values, one for the foreground and one for the background, although
the background color can be specified as {\em transparent}, in which case
background pixels are left unchanged.
\item[Copy a rectangular image]
This operation is often called a {\em rasterop}.
The application needs to specify the top left hand corner and the width and height
of the area to be copied as well as the coordinates of the destination
area. 
There are two {\em wn} rasterop functions: one intended for color
displays which simply copies the source on top of the destination,
and one intended for monochrome displays which allows the source
and destination pixel values to be combined logically.
On a monochrome display the pixels can be thought of as having the logical
values {\em true} and {\em false} representing foreground and background
respectively
and the source and destination can be combined using logical operators
like {\bf and} and {\bf or}.
\item[Painting a rectangle]
The rectangle's top left hand corner, width and height need to be specified
along with the new pixel value.
Painting an area with the background color is used to clear a window, erase
displayed text etc. If the intention is to highlight an area of the screen then
a grey pattern of set pixels can be used as shading.
\end{description}
%%%%%%%%%%%%%%%%%%%%%%%%%%%%%%%%%%%%%%%%
\subsection{Reading input}
Since input to an interactive workstation program comes from two distinct
physical sources (keyboard and mouse), it is presented to the program
as a series of events. An input event is represented by a C structure
with fields describing the cause of the event (called the event type)
and various event parameters.
Among the possible causes of an event are:
\begin{itemize}
\item A mouse button was pressed.
\item A keyboard key was pressed.
\item The mouse has moved.
\item The window size has changed and so it needs redrawing.
\end{itemize}
Not every event parameter is valid for every event type.
Some typical event parameters are:
\begin{itemize}
\item The cursor coordinates at the time of the event.
\item the state of the mouse buttons, i.e. which ones are up and which are down.
\item The keyboard character if the event was a key press.
\end{itemize}
The {\em wn} library function which obtains the next event also takes
an argument which can be used to specify which event types are of interest.
Hence if a program does not need to know about mouse movement but only
needs to know the mouse position when a button is pressed then it can
ask not to be told about mouse movement events.
Similarly a program which does not read the keyboard can elect not
to be informed of keyboard events.
%%%%%%%%%%%%%%%%%%%%%%%%%%%%%%%%%%%%%%%%
\subsection{Setting the cursor}
The cursor is a small binary image which tracks the mouse and is
used for pointing at and selecting objects from the display.
The displayed cursor image can be changed at any time (see section \ref{setcurs});
changing the cursor is a useful way of providing the user with feedback.
%%%%%%%%%%%%%%%%%%%%%%%%%%%%%%%%%%%%%%%%
\subsection{Special cursors}
{\em Wn} provides a selection of cursor-tracking rubber lines, rubber boxes
and cross-hairs. These devices, which are collectively known as {\em special cursors},
are discussed in detail in section \ref{spcurs}.
%%%%%%%%%%%%%%%%%%%%%%%%%%%%%%%%%%%%%%%%
\section{Using the wn library}
In every file that uses routines in this library you should have the
line
\begin{verbatim}
#include <local/wn.h>
\end{verbatim}
near the beginning.
This includes a header file which defines various structures
and constants used by {\em wn}.

You should link programs against the {\em wn} library by
including the argument {\tt -l}{\it $<$windowsystem$>$}{\tt wn} in the loader
command line, where {\it $<$windowsystem$>$} depends on the window
system you are using.
This argument should appear after your `.o' files but before any
machine dependent graphical libraries.
The following are example linking commands for the window systems that
{\em wn} runs over:

\begin{tabular}{ll}

{\bf X11:}&{\tt cc <options> <your .o files> -lx11wn -lX11}\\
{\bf X10:}&{\tt cc <options> <your .o files> -lx10wn -lX10}\\[0.1in]

{\bf Sunview:}&{\tt cc <options> <your .o files> -lsunviewwn -lsunwindow -lpixrect}\\
\end{tabular}
%%%%%%%%%%%%%%%%%%%%%%%%%%%%%%%%%%%%%%%%
\section{Windows}
A window used by {\em wn} is identified by a {\em window number} which
is a small positive integer.
Most {\em wn} functions need a window number as an argument.
An application normally has a single top level window, although you can
create more than one top level window with {\em wn}.
A top level window is a fairly heavyweight object -- creating one is
usually relatively slow.
\subsection {Top level windows}
The function:
\funclabel{wn_open_stdwin}
\begin{verbatim}
int wn_open_stdwin()
\end{verbatim}
opens the `standard window' and returns its window number.
Under some window managers the created window will be the one the application was started
in; on others, the user will be prompted for a window.
In any case, the usual local  conventions for creating the main window of a
graphical program will be followed.

{\tt Wn\_open\_stdwin} returns $-1$ if the window cannot be created (usually
because the window manager is not running).
If the window is opened successfully, it returns the window number of
the opened window and this value is used to specify the window by passing it as the
{\tt wn} argument to
any graphical input or output routine.

\funclabel{wn_create_window}
\begin{verbatim}
int wn_create_window()
\end{verbatim}
This function is similar to {\tt wn\_open\_stdwin}; it creates a new
top level window and returns a window number for it.
Applications which use more than one top level windows should use this
function to create windows other than their main one.
%%%%%%%%%%%%%%%%%%%%%%%%%%%%%%%%%%%%%%%%
\subsection{Obtaining information about a window}
The size of a window can be obtained using the function
\funclabel{wn_get_window_size}
\begin{verbatim}
void wn_get_window_size(wn, p_width, p_height)
int wn;
int *p_width, *p_height;
\end{verbatim}
which sets {\tt *p\_width} and {\tt *p\_height} to the current width
and height of window {\tt wn} in pixels.

The functions {\tt wn\_get\_width} and {\tt wn\_get\_height} return the
current width and height respectively of the window:
\funclabel{wn_get_width}
\begin{verbatim}
int wn_get_width(wn)
int wn;
\end{verbatim}
\funclabel{wn_get_height}
\begin{verbatim}
int wn_get_height(wn)
int wn;
\end{verbatim}

The function {\tt wn\_is\_open} can be used to check whether a window
number refers to an open window:
\funclabel{wn_is_open}
\begin{verbatim}
int wn_is_open(wn)
int wn;
\end{verbatim}
{\tt Wn\_is\_open} returns non zero if window {\tt wn} is open.

A program which takes advantage of color when available but also
runs on a monochrome display needs to determine whether the
current display is monochrome or color. The function
\funclabel{wn_get_nplanes}
\begin{verbatim}
int wn_get_nplanes()
\end{verbatim}
\sloppy{returns the number of planes supported by the current display.
{\tt Wn\_get\_nplanes} is also needed in order to determine the amount
of memory needed to store an off-screen copy of an image
(see section \ref{image}).}

A program which needs to know which windowing system it is running under can
find out using the function:
\funclabel{wn_get_wm_type}
\begin{verbatim}
int wn_get_wm_type()
\end{verbatim}
The value returned is one of:
\begin{center}
\begin{tabular}{|l|p{2in}|} \hline
{\tt WN\_SUNVIEW} & Sun running SunView \\ \hline
{\tt WN\_X10} & X version 10 \\ \hline
{\tt WN\_X11} & X version 11 \\ \hline
\end{tabular}\\
\end{center}
It should be stressed that most programs which use {\em wn} do not need to
know which window system they are running under and so never need to call
{\tt wn\_get\_wm\_type}.
\subsection{Closing on windows}
The function
\funclabel{wn_close_window}
\begin{verbatim}
void wn_close_window(wn)
int wn;
\end{verbatim}
closes window {\tt wn}, and does any tidying up needed if {\tt wn}
is a top level window.
Any subwindows of {\tt wn} are also closed.
This function must be called for any open top level windows before exiting.
Failure to do so can leave unusable windows on the screen.
%%%%%%%%%%%%%%%%%%%%%%%%%%%%%%%%%%%%%%%%
\section{Graphical output}
{\em Wn} supports four kinds of graphical output: line, text, rectangle and rasterop.
Each of the first three drawing functions takes a pixel value argument (two pixel values
for text). {\em Wn} provides two pre-defined pixel values {\tt WN\_FG} and {\tt WN\_BG}
representing the foreground and background colors.
These should be used by monochrome software which might be
run on a monochrome display or 
color display. On a monochrome display {\tt WN\_FG} is usually black and {\tt WN\_BG}
is white whereas on a color display they will have preset values which depend
on the windowing environment. On a color display
running X the colors represented by {\tt WN\_FG} and {\tt WN\_BG} can be set using
the standard command line arguments or defaults database entries
(see section \ref{munge_args}).
%%%%%%%%%%%%%%%%%%%%%%%%%%%%%%%%%%%%%%%%
\pagebreak[1]
\subsection{Lines}
\funclabel{wn_draw_line}
\begin{verbatim}
void wn_draw_line(wn, x1, y1, x2, y2, value)
int wn;
int x1, y1, x2, y2, value;
\end{verbatim}
{\tt Wn\_draw\_line} draws a single pixel wide line from
{\tt x1}, {\tt y1} to {\tt x2}, {\tt y2} 
using {\tt value} as the pixel value. When working in monochrome the globally
defined pixel value {\tt WN\_FG} can be used to draw a line in the foreground
color. When working in color {\tt value} should be a pixel value whose colormap
entry has been allocated and initialised by the application
(see section \ref{color}).
%%%%%%%%%%%%%%%%%%%%%%%%%%%%%%%%%%%%%%%%
\subsection{Text}
\funclabel{wn_text}
\begin{verbatim}
void wn_text(wn, font, s, x, y, fg_value, bg_value, use_which)
int wn;
font_t *font;
char *s;
int x, y, value, use_which;
\end{verbatim}
{\tt Wn\_text} prints the NUL terminated string {\tt s} in font {\tt font}
at position {\tt x},{\tt y} in window {\tt wn}.
{\tt Font} is a {\tt font\_t} structure as described in the section on fonts
(section \ref{fonts}); using a {\tt NULL} pointer for {\tt font} will cause
the system font to be used. Alternatively there are several predefined macros
which provide an easy way of displaying text in the system font
(see section \ref{othergraph}).
The system font will by default be a standard font built into the windowing
system but most windowing systems provide some way for the user to specify the
system font associated with a particular window (see section \ref{munge_args} for
an explanation of the way this is done under X).

Set bits in a character image are drawn using pixel value {\tt fg\_value},
clear bits using {\tt bg\_value}.
The special background value {\tt WN\_TRANSPARENT} specifies that clear pixels
in the font should not modify the display.
This is sometimes useful for color displays but usually involves a high processing
overhead.

{\tt Use\_which} should be {\tt WN\_USE\_BASELINE} or {\tt WN\_USE\_TOP}\@.
{\tt WN\_USE\_BASELINE} specifies that the {\tt y} coordinate given
is where the baseline of the text printed should appear.
{\tt WN\_USE\_TOP} says that the given
{\tt y} coordinate is where the top of the displayed text should appear.
%%%%%%%%%%%%%%%%%%%%%%%%%%%%%%%%%%%%%%%%
\subsection{Painting a rectangle}
\funclabel{wn_set_area}
\begin{verbatim}
void wn_set_area(wn, x, y, width, height, value)
int wn;
int x, y, width, height, value;
\end{verbatim}
The value of each of the pixels in the given rectangle is set to {\tt value}.
Even for monochrome software, {\tt wn\_set\_area} is preferable to rasterop
as a way to clear ({\tt value == WN\_BG}) or set ({\tt value == WN\_FG}) areas
of the window, as it will work on color displays.
%%%%%%%%%%%%%%%%%%%%%%%%%%%%%%%%%%%%%%%%
\subsection{Rasterops}
\funclabel{wn_rop}
\begin{verbatim}
void wn_rop(wn, src_x, src_y, width, height, dst_x, dst_y)
int wn;
int src_x, src_y, width, height, dst_x, dst_y;
\end{verbatim}
Copies the area in window {\tt wn} given by {\tt src\_x, src\_y, width, height}
to {\tt dst\_x, dst\_y}.

Under the X window system, this operation can fail if any part of the source
area is not visible on the screen (i.e. off the edge of the screen or
obscured by another window).
You can check for this with the {\tt wn\_last\_rop\_was\_damaged} function:
\funclabel{wn_last_rop_was_damaged}
\begin{verbatim}
int wn_last_rop_was_damaged(wn)
int wn;
\end{verbatim}
This returns true if this happened the last time {\tt wn\_rop} was called
on window {\tt wn}.
If this happens the application should repaint the destination area.
The easiest way to do this is to repaint the entire window as if
the window had been exposed, but this usually looks messy.
It is better just to repaint the destination area if possible.
%%%%%%%%%%%%%%%%%%%%%%%%%%%%%%%%%%%%%%%%
\subsection{Other graphical output functions}
\label{othergraph}
{\em Wn} provides a few other graphical output functions.
These are built on the routines described above.
\funclabel{wn_move_area}
\begin{verbatim}
void wn_move_area(wn, src_x, src_y, width, height, dst_x, dst_y, value)
int wn;
int src_x, src_y, width, height, dst_x, dst_y, value;
\end{verbatim}
{\tt Wn\_move\_area} is useful for scrolling; it
copies the area in window {\tt wn} given by {\tt src\_x, src\_y, width, height}
to {\tt dst\_x, dst\_y}, and sets pixels in the area vacated to {\tt value}.
The current version of {\tt wn\_move\_area} is restricted to horizontal
or vertical motion.
As with {\tt wn\_rop}, you should check for damage with
{\tt wn\_last\_rop\_was\_damaged}.
\funclabel{wn_box_round}
\begin{verbatim}
void wn_box_round(wn, x, y, width, height, value)
int wn;
int x, y, width, height, value;
\end{verbatim}
A rectangle is drawn round the given area using the pixel value given.
Note that the rectangle drawn surrounds the area -- the top left
hand corner is at {\tt x-1}, {\tt y-1}.

The following four functions are simpler than {\tt wn\_text} if only the system
font is being used.
\funclabel{wn_btext}
\begin{verbatim}
void wn_btext(wn, s, x, y, fg_value, bg_value)
int wn;
char *s;
int x, y;
int fg_value, bg_value;
\end{verbatim}
\funclabel{wn_ttext}
\begin{verbatim}
void wn_ttext(wn, s, x, y, fg_value, bg_value)
int wn;
char *s;
int x, y;
int fg_value, bg_value;
\end{verbatim}
\funclabel{wn_bputs}
\begin{verbatim}
void wn_bputs(wn, s, x, y)
int wn;
char *s;
int x, y;
\end{verbatim}
\funclabel{wn_tputs}
\begin{verbatim}
void wn_tputs(wn, s, x, y)
int wn;
char *s;
int x, y;
\end{verbatim}

{\tt Wn\_btext} uses {\tt WN\_USE\_BASELINE} as the {\tt use\_which} argument
so that the {\tt y} coordinate is that of the baseline.
{\tt Wn\_ttext} uses {\tt WN\_USE\_TOP}, so the {\tt y} coordinate is that
of the top left pixel of the first character.

{\tt Wn\_tputs} and {\tt wn\_bputs} correspond to {\tt wn\_ttext} and
{\tt wn\_btext}, but default the foreground and background pixels to
{\tt WN\_FG} and {\tt WN\_BG}.

%%%%%%%%%%%%%%%%%%%%%%%%%%%%%%%%%%%%%%%%
\subsection{Additional functions available for monochrome displays}
If only two colors are being used then it is possible to regard the display as a
rectangular array of bits which can be set and cleared.
In particular rasterops can be used to do things like clearing or
inverting an area.

These techniques do not in general work on color displays,
as a pixel has a value rather than simply being a set or clear bit.
Using logical functions on a pixel value will quite likely set the
pixel to some value for which the colormap entry is undefined, and
unpredictable colors will be displayed.

{\em Wn} provides some functions to allow applications which treat the
display as an array of bits to run on monochrome or color displays.
So long as the only pixel values used are {\tt WN\_FG} and {\tt WN\_BG} the
display behaves as if it were monochrome -- you can use {\bf xor} rasterops to
clear areas and do other similar tricks.

The effect of these functions on pixels with values other than {\tt WN\_FG}
and {\tt WN\_BG} is undefined --- the pixels will usually be set to some
random value (and hence color).
\funclabel{wn_mono_rop}
\begin{verbatim}
void wn_mono_rop(wn, sx, sy, width, height, dx, dy, ropfunc)
int wn;
int sx, sy, width, height, dx, dy, ropfunc;
\end{verbatim}
Like {\tt wn\_rop} above, except that the rasterop acts on foreground and
background pixels as if the pixel values were 1 and 0 respectively.
As with {\tt wn\_rop}, you should check for damage with
{\tt wn\_last\_rop\_was\_damaged}.

The {\tt ropfunc} argument determines how the source and destination
are combined. The available values are:
\pagebreak[3]
\begin{verbatim}
R_RPL           Dest = Source
R_NOT           Dest = NOT Source
R_AND           Dest = Dest AND Source
R_ANDNOT        Dest = Dest AND NOT Source
R_OR            Dest = Dest OR Source
R_ORNOT         Dest = Dest OR NOT Source
R_XOR           Dest = Dest XOR Source
R_XNOR          Dest = Dest XOR NOT Source
\end{verbatim}
Of these functions, only {\tt R\_RPL} will have predictable results on color
displays.
The other functions are useful with single plane bitmaps.

The following function can be used to invert all the pixels in a rectangle:
\funclabel{wn_invert_area}
\begin{verbatim}
void wn_invert_area(wn, x, y, width, height)
int wn;
int x, y, width, height;
\end{verbatim}
{\tt Wn\_invert\_area} swaps the values of all the
foreground ({\tt WN\_FG}) and background ({\tt WN\_BG}) pixels.
A second call specifying the same area will restore all pixels
to their former values.
On color displays, pixels with values other than {\tt WN\_FG} or {\tt WN\_BG}
will be set to unpredictable colors.

There is also a function to invert a hollow rectangle of pixels:
\funclabel{wn_invert_box}
\begin{verbatim}
void wn_invert_box(wn, x, y, width, height)
int wn;
int x, y, width, height;
\end{verbatim}
This function has the same effect as {\tt wn\_box\_round}
(see section \ref{wn_box_round}) except that
it inverts pixels rather than setting them to a pixel value.

When working in monochrome it is often useful to be able to highlight an area of
the display by shading the background. The {\em wn} function to do this is:
\funclabel{wn_shade_area}
\begin{verbatim}
void wn_shade_area(wn, x, y, width, height, shade, ropfunc)
int wn;
int x, y, width, height, ropfunc;
short shade;
\end{verbatim}
{\tt Shade} is interpreted as a 4 by 4 rectangle of pixels which is painted
into the given area using the rasterop function {\tt ropfunc}.
There are a number of predefined patterns
named {\tt WN\_GREY2}, {\tt WN\_GREY4}, \ldots, {\tt WN\_GREY14} in which the numerical
part of the name is the number of set bits and thus the density of the grey shading.
The combination of shading density and rasterop function will depend on the application;
it is sometimes convenient to shade using {\tt R\_XOR} because the effect can be removed
simply by repeating the operation but this will only work with a very light or very dense shade.
As with other graphical functions having a rasterop argument, the only safe rasterop
function to use in conjunction with color is {\tt R\_RPL}.

{\tt Wn\_bold} provides a crude way of highlighting a piece of displayed text:
\funclabel{wn_bold}
\begin{verbatim}
void wn_bold(wn, x, y, width, height)
int wn;
int x, y, width, height;
\end{verbatim}
The given area is combined with itself shifted one pixel to the right,
using rasterop function {\tt R\_OR}.
{\tt Wn\_bold} has the effect of emboldening text in the area if the text is displayed
with color {\tt WN\_FG}.
The effect on pixels with values other than {\tt WN\_FG} or {\tt WN\_BG} is
unpredictable.
\funclabel{wn_mono_text}
\begin{verbatim}
void wn_mono_text(wn, font, s, x, y, ropfunc, use_which)
int wn;
font_t font;
char *s;
int x, y, ropfunc, use_which;
\end{verbatim}
Monochrome version of {\tt wn\_text}.
Pixels are drawn using the foreground pixel value ({\tt WN\_FG}),
using rasterop function {\tt ropfunc}.
Code which does things like using {\tt R\_NOT} to draw inverted text should
use this function.
\funclabel{wn_invert_line}
\begin{verbatim}
void wn_invert_line(wn, x1, y1, x2, y2)
int wn;
int x1, x2, y1, y2;
\end{verbatim}
Like {\tt wn\_draw\_line}, but it inverts the pixels under the line.
That is, {\tt WN\_FG} pixels become {\tt WN\_BG} pixels and vice versa.
On color displays, pixels with values other than {\tt WN\_FG} or
{\tt WN\_BG} have unpredictable values after this call.

A second {\tt wn\_invert\_line} with the same coordinates will restore all
pixels to their former values, even on color displays.

%%%%%%%%%%%%%%%%%%%%%%%%%%%%%%%%%%%%%%%%
\subsection{Batching of window output}
A series of graphical output requests are said to be batched
if none of them are executed until the last one has been sent.
Ideally this should result in the intermediate, partially
redrawn, states not being displayed (so that, for example, an erase followed by
a redrawing with minor changes leaves the unchanged parts of the display
continuously displayed).
Unfortunately this is not possible under current versions of X.
Batching usually produces a performance improvement
and so it is still worth batching a series of related graphical
output requests even if the underlying window system does not suppress intermediate
redrawing.
A sequence of graphical function calls are batched by bracketing
them by the functions:
\label{wn_updating_off}
\begin{verbatim}
wn_updating_off(wn)
int wn;
\end{verbatim}
\funclabel{wn_updating_on}
\begin{verbatim}
wn_updating_on(wn)
int wn;
\end{verbatim}
A typical sequence of calls is:

\hspace{0.5in}{\tt wn\_updating\_off(wn);}

\hspace{0.5in}{\em lots of graphical output to the window}

\hspace{0.5in}{\tt wn\_updating\_on(wn);}

The output batching functions can be nested.
This means that only the first of a set of nested of calls
to {\tt wn\_updating\_off} turns off updating, and only the last call
of {\tt wn\_updating\_on} (the one corresponding to the outermost call of
{\tt wn\_updating\_off}) turns updating on again.

Occasionally you want to show the latest updates to the window without
affecting the updating state. The function
\funclabel{wn_show_updates}
\begin{verbatim}
void wn_show_updates(wn)
int wn;
\end{verbatim}
refreshes the window (shows the latest changes) without affecting the
updating level.
%%%%%%%%%%%%%%%%%%%%%%%%%%%%%%%%%%%%%%%%
\subsection{Outputting a beep}
\funclabel{wn_bell}
\begin{verbatim}
void wn_bell(wn)
int wn;
\end{verbatim}
Makes a beep.
On some machines the `beep' may be visual, hence the need for the
{\tt wn} argument.
%%%%%%%%%%%%%%%%%%%%%%%%%%%%%%%%%%%%%%%%
\section{Input}
{\em Wn} formats input into a stream of events.
The input routine {\tt wn\_next\_event} gets the next event for a window.
%%%%%%%%%%%%%%%%%%%%%%%%%%%%%%%%%%%%%%%%
\subsection{Events}
The event types are:
\begin{center}
\begin{tabular}{|l|p{4in}|} \hline
{\tt EV\_KEY} & Keyboard key pressed \\ \hline
{\tt EV\_BUTTON\_DOWN} & Mouse button pressed \\ \hline
{\tt EV\_BUTTON\_UP} & Mouse button released \\ \hline
{\tt EV\_MOUSE\_MOVED} & Mouse moved.
There is no guarantee that the returned coordinates have changed since the
last event. \\ \hline
{\tt EV\_WINDOW\_RESIZED} & The window's size has been changed. \\ \hline
{\tt EV\_WINDOW\_EXPOSED} & The window's size has not changed, but it needs
repainting (this event never happens on some machines).
Most applications can treat it in the same way as {\tt EV\_WINDOW\_RESIZED}. \\ \hline
{\tt EV\_WINDOW\_SELECTED} & The user selected the window. \\ \hline
{\tt EV\_WINDOW\_DESELECTED} & The user deselected the window.
Most applications can ignore this and {\tt EV\_WINDOW\_SELECTED}. \\ \hline
{\tt EV\_INTERRUPT} & A signal (such as {\tt SIGALRM}) occurred. \\ \hline
{\tt EV\_OTHER\_INPUT} &
Input is pending on a user specified file descriptor -- see the description
of multiplexed input below. \\ \hline
{\tt EV\_OTHER} & An event that {\em wn} did not recognise was generated by the machine. \\ \hline
\end{tabular}\\[0.1in]
{\em Wn event types}
\end{center}
The structure defining an event is:
\begin{verbatim}
typedef struct {
        unsigned long ev_type;
        char ev_flags;
        char ev_char;
        short ev_x;
        short ev_y;
        short ev_buttons;
        long ev_time;
        int ev_wn;
        int ev_fdmask;
} event_t;
\end{verbatim}
The meanings of the fields are as follows:
\begin{center}
\begin{tabular}{|l|p{4.75in}|} \hline
{\tt ev\_type} & the type of this event --- see section \ref{evtype}. \\ \hline

{\tt ev\_flags} & flag bits for this event.
For {\tt EV\_BUTTON\_UP} and {\tt EV\_BUTTON\_DOWN} events, {\tt ev\_flags}
has exactly one of {\tt B\_LEFT}, {\tt B\_MIDDLE} or
{\tt B\_RIGHT} set, saying which mouse button went up or down.
If the event was pushed back (see {\tt wn\_pushback\_event} below), the
{\tt EV\_WAS\_PUSHED\_BACK} bit is set. \\ \hline

{\tt ev\_char} &
On {\tt EV\_KEY} events, the ASCII code of the key that was pressed.
This field is only valid on {\tt EV\_KEY} events. \\ \hline

{\tt ev\_x, ev\_y} &
the latest coordinates of the mouse,
relative to the top left corner of the window.
They are valid on all events. \\ \hline

{\tt ev\_buttons} &
a bit mask representing the current state of the mouse buttons.
Zero or more of the button bit values {\tt B\_LEFT}, {\tt B\_MIDDLE} and
{\tt B\_RIGHT} may be set.
Like {\tt ev\_x} and {\tt ev\_y}, this field is valid for all events. \\ \hline

{\tt ev\_time} &
a timestamp, saying when the event happened.
[ Not yet implemented ]. \\ \hline

{\tt ev\_fdmask} &
On {\tt WN\_OTHER\_INPUT} events,
a bitmap of the file descriptors for which input is pending.
See the description of multiplexed input below. \\ \hline

{\tt ev\_wn} &
the window the event occurred in.
This is useful when you ask for an event from any window. \\ \hline
\end{tabular}\\[0.1in]
{\em Wn event structure fields}
\end{center}
%%%%%%%%%%%%%%%%%%%%%%%%%%%%%%%%%%%%%%%%
\subsection{Event types}
\label{evtype}
The event types listed above ({\tt EV\_KEY}, {\tt EV\_BUTTON\_DOWN}, etc) are
used in two contexts:

In the {\tt ev\_type} field of a returned event, only one of the bits is
ever set, indicating which event occurred.
Normally in an event handling routine the application will switch on
{\tt ev\_type}, with the {\tt EV\_*} event types as the case labels.

In the mask passed to {\tt wn\_next\_event} (below) one or more of the
event types are ORed together to form a mask, indicating which events are
of interest.
%%%%%%%%%%%%%%%%%%%%%%%%%%%%%%%%%%%%%%%%
\subsection{Getting events}
\funclabel{wn_next_event}
\begin{verbatim}
void wn_next_event(wn, mask, event)
int wn, mask;
event_t *event;
\end{verbatim}
Fills in {\tt event} with the next event for window {\tt wn} whose
type matched {\tt mask}.
If {\tt wn} is {\tt WN\_ANY}, the next event from any of the currently
open windows or subwindows is generated (see section \ref{events_and_subwindows}
below).

{\tt Mask} gives the events of interest, and is constructed by
ORing together the event codes of the events of interest.
Events that do not match {\tt mask} are discarded.

\funclabel{wn_get_resize_event}
\begin{verbatim}
int wn_get_resize_event(wn)
int wn;
\end{verbatim}
This routine returns {\tt EV\_WINDOW\_RESIZED} if there have been one or
more resize or expose events for window {\tt wn} since the last call of
this routine on that window.
If there have been no resize events but one or more expose
event have occurred, then {\tt EV\_WINDOW\_EXPOSED} is returned.
If there have been no resize or expose events zero is returned.

This routine is useful if you don't want to check for window size changes
on every event.
The application can just call {\tt wn\_get\_resize\_event} at suitably
frequent intervals.
\funclabel{wn_pushback_event}
\begin{verbatim}
void wn_pushback_event(event)
event_t *event;
\end{verbatim}
Arranges that the next call of {\tt wn\_next\_event} will return an
event identical to {\tt event}, except that the {\tt EV\_WAS\_PUSHED\_BACK}
bit will be set in {\tt ev\_flags}.
This is useful when you read an event and then decide that you want to process it
later.
%%%%%%%%%%%%%%%%%%%%%%%%%%%%%%%%%%%%%%%%
\subsection{Events and Subwindows}
\label{events_and_subwindows}
Many applications have a main window that is divided into a number of
rectangular regions --- a menu, one or more scroll bars, a main display
area, etc.
The first task in such applications when an event is received is to
determine which region of the main window the mouse was in when the
event occurred.

One way of doing this is to make each region of the window a
subwindow (see section \ref{subwindows}), and to give a window
number of {\tt WN\_ANY} to {\tt wn\_next\_event}.
In this case the {\tt ev\_wn} field of the event will be set to the
smallest window or subwindow that contains the mouse.
The application can use this to determine which region of the main
window the event occurred in.

{\em Wn} provides a way of directly converting from a window number
to a application defined integer or pointer.
The function:
\funclabel{wn_set_win_data}
\begin{verbatim}
long wn_set_win_data(wn, data)
int wn;
long data;
\end{verbatim}
associates {\tt data} with window {\tt wn} and returns the previous
value of the data (zero if no value has been set).

The data stored can be retrieved with the function:
\funclabel{wn_get_win_data}
\begin{verbatim}
long wn_get_win_data(wn)
int wn;
\end{verbatim}
This returns the value previously stored by {\tt wn\_set\_win\_data}.

Normally the application will set up a data structure representing
the various regions of the main window.
It will then create subwindows for each region, and use {\tt wn\_set\_win\_data}
to store a pointer to the structure describing the region with its
subwindow.
It then passes the window number {\tt WN\_ANY} to {\tt wn\_next\_event}
and uses {\tt wn\_get\_win\_data} to convert from the window number
of the event ({\tt ev\_wn}) to a pointer the structure describing the region.
%%%%%%%%%%%%%%%%%%%%%%%%%%%%%%%%%%%%%%%%
\subsection{Setting the input mode}
{\tt Wn\_next\_event} usually blocks when there is no input pending ---
that is, it does not return until an event arrives.
This is nearly always the correct behaviour, but it is sometimes
necessary to do input in non blocking mode. The function:
\funclabel{wn_inmode}
\begin{verbatim}
void wn_inmode(wn, mode)
int wn, mode;
\end{verbatim}
sets the input mode for window {\tt wn} to {\tt mode}, which should be
{\tt WN\_SAMPLE}, {\tt WN\_REQUEST}, or {\tt WN\_NOINPUT}.
{\tt Wn\_inmode} returns the previous input mode.

In sample mode, {\tt wn\_next\_event} will always return immediately 
(unless the window is deselected, when it will block until the window
is selected again).
If there are no events pending for the window, 
an {\tt EV\_MOUSE\_MOVED} event is returned,
even though the mouse coordinates will not in fact have changed.

{\tt WN\_NOINPUT} mode is useful when you will be busy processing for
some time and hence ignoring events.
All events are discarded until a subsequent change to sample or request
mode.
The X servers kill a client if too many events are queued up so you
should go into {\tt WN\_NOINPUT} mode whenever you will be ignoring
events for a long time.
%%%%%%%%%%%%%%%%%%%%%%%%%%%%%%%%%%%%%%%%
\subsection{Multiplexed input}
On some machines it is possible to check for input on many file descriptors
with a single system call (see {\tt select(2)}).
On these machines, {\em wn} gives you some access to this capability with the
{\tt EV\_OTHER\_INPUT} event.
When this event is returned, the {\tt ev\_fdmask} event field is a bit mask,
with bit {\tt n} set if there is input pending on file descriptor {\tt n}.

{\em Wn} maintains internally a mask of file descriptors which could have
input pending; this mask is initially zero.
Two functions are defined to let you examine and set this mask:
\funclabel{wn_get_fd_mask}
\begin{verbatim}
int wn_get_fd_mask()
\end{verbatim}
\funclabel{wn_set_fd_mask}
\begin{verbatim}
void wn_set_fd_mask(mask)
int mask;
\end{verbatim}
To add a file descriptor {\tt fd} to the mask, use:
\begin{verbatim}
wn_set_fd_mask(wn_get_fd_mask() | (1 << fd));
\end{verbatim}
To remove a file descriptor, use:
\begin{verbatim}
wn_set_fd_mask(wn_get_fd_mask() & ~(1 << fd));
\end{verbatim}

If you add a non valid file descriptor to the mask, or a file descriptor in the
mask becomes invalid, the next call of {\tt wn\_next\_event} will abort with
an error message.
The application must remove file descriptors from the mask {\em before}
closing them.

It is important to note that this facility is only available on machines
which have the {\tt select(2)} system call, or something similar in
functionality.

On machines without {\tt select}, no {\tt EV\_OTHER\_INPUT} events are ever
generated, and {\tt wn\_set\_fd\_mask} writes a warning message to the
standard error whenever it is called.

Some applications may want to do their own {\tt select} call.
To allow this style of input, there is a function to obtain a mask
containing all file descriptors used by {\em wn}:
\funclabel{wn_get_wm_fds}
\begin{verbatim}
int wn_get_wm_fds()
\end{verbatim}
This mask can be {\em or}ed with the application's file descriptors to
form a mask for {\tt select}.
In this way an application can check itself whether there is
graphical input pending.
%%%%%%%%%%%%%%%%%%%%%%%%%%%%%%%%%%%%%%%%
\subsection{Other event routines}
There are a few routines built on top of {\tt wn\_next\_event}:
\funclabel{wn_wait_for_release_of}
\begin{verbatim}
void wn_wait_for_release_of(wn, button)
int wn, button;
\end{verbatim}
This repeatedly calls {\tt wn\_next\_event} until the button specified
in {\tt button} is no longer pressed.
Multiple buttons can be specified by ORing together button masks.
\funclabel{wn_await_window_size_change}
\begin{verbatim}
void wn_await_window_size_change(wn)
int wn;
\end{verbatim}
This waits for an {\tt EV\_WINDOW\_RESIZED} event from window {\tt wn}.
Useful if a window has been made too small for the application to run
in.
This calls {\tt wn\_next\_event} with just {\tt EV\_WINDOW\_RESIZE}
in the event mask, so all other events on the window are discarded until
the resize event.
%%%%%%%%%%%%%%%%%%%%%%%%%%%%%%%%%%%%%%%%
\subsection{Asynchronous input}
\funclabel{wn_set_abort_func}
\begin{verbatim}
int (*wn_set_abort_func(wn, func))()
int wn;
int (*func)();
\end{verbatim}
A graphical program often needs to provide some way for the user to
interrupt long running commands. {\tt Wn\_set\_abort\_func}
sets the user abort function for window {\tt wn} to {\tt func}.
This arranges that the function {\tt func} will be called asynchronously
when the user presses an interrupt key.
When running under SunView typing the {\bf stop} key (conventionally the
key labeled {\tt L1}) into the graphical window will
cause {\tt func} to be called. When running under X a program can be
interrupted by typing CONTROL-C into the window from which the program was
started. This is intended as a stop-gap and will be replaced by something
better in the future.

The arrangement is cancelled by calling {\tt wn\_set\_abort\_func} with
a {\tt NULL} {\tt func}.

You should probably go into {\tt WN\_NOINPUT} mode (see section \ref{wn_inmode})
before setting up the handler, and back to request mode after clearing it.
Even if you don't do this, you should not call any input routines
while the interrupt handler is set up.
\subsection{Warping the mouse}
In certain rare situations it is useful for the application to move the
cursor itself.  This is known as {\em warping} the mouse.
Doing this is almost always a bad idea, but in the same way that C
provides {\tt goto}, {\em wn} provides a function to warp the mouse:
\funclabel{wn_warp_mouse}
\begin{verbatim}
void wn_warp_mouse(wn, x, y)
int wn;
int x, y;
\end{verbatim}
Moves the cursor to {\tt x}, {\tt y} in window {\tt wn}.
%%%%%%%%%%%%%%%%%%%%%%%%%%%%%%%%%%%%%%%%
\section{Using color displays}
Most graphical output routines take a {\tt value} argument which specifies
the pixel value to be used for modified pixels. The workstation display
hardware then uses a {\it colormap} table to convert the pixel value
into a true color which it displays on the screen. It follows from this that
a program which wants to produce color images must set up the colormap
entries it needs before it attempts to draw on the screen.
Since there are a fixed number (usually 256) of available pixel values
and there may be
several programs each wanting to display a color image in its own window, the
pixel values and their corresponding colormap entries are treated by
{\em wn} as a shareable resource.

A program which produces a monochrome display but which wants to run consistently
on monochrome and color displays should use the two pre-assigned pixel values
{\tt WN\_FG} and {\tt WN\_BG} for its foreground and background colors.  A program
which wants to use more than two colors must request a set of pixel values from {\em wn}
and install the corresponding colormap entries.

Some applications simply want a given set of colors, and don't require
any relationship between the pixel values of the colors.
{\em Wn} provides a set of basic colormap allocation functions for this.
Other applications need to be able to do logical operations on already
displayed pixels which result in other pixel values.
These applications use a more sophisticated colormap allocation
routine.
%%%%%%%%%%%%%%%%%%%%%%%%%%%%%%%%%%%%%%%%
\subsection{Colormap entries}
\label{color}
The colormap functions operate on arrays of the following structure:
\begin{verbatim}
typedef struct {
        short co_pixel;
        unsigned short co_red;
        unsigned short co_green;
        unsigned short co_blue;
} color_t;
\end{verbatim}
\begin{list}{}{}
\item[\tt co\_pixel]
specifies the pixel value associated with the color definition
\item[{\tt co\_red, co\_green} and  {\tt co\_blue}]
give the levels of red, green and blue for this color.
The values range from 0 (off) to 65535 (full brightness).
Thus (0, 0, 0) is black and (65535, 65535, 65535) is white.
In practice only the top byte of each color level is significant since available
display hardware only provides 256 different intensities for each primary color.
\end{list}
%%%%%%%%%%%%%%%%%%%%%%%%%%%%%%%%%
\subsection {Simple colormap manipulation}
The following functions are used to request a set of pixel values and
set their colormap entries.
\funclabel{wn_get_pixels}
\begin{verbatim}
int wn_get_pixels(colors, ncolors)
color_t *colors;
int ncolors;
\end{verbatim}
\funclabel{wn_set_pixel_colors}
\begin{verbatim}
void wn_set_pixel_colors(colors, ncolors)
color_t *colors;
int ncolors;
\end{verbatim}
\funclabel{wn_get_pixels_by_color}
\begin{verbatim}
int wn_get_pixels_by_color(colors, ncolors)
color_t *colors;
int ncolors;
\end{verbatim}
The function {\tt wn\_get\_pixels}
allocates {\tt ncolors} colormap cells, and sets the {\tt co\_pixel} field
of each element of the {\tt colors} array to the allocated pixel value.
The {\tt co\_red}, {\tt co\_green} and {\tt co\_blue} fields are ignored and
are not altered; the colors of the pixels are undefined on return.
This routine will fail if there are not enough free colormap entries ---
it returns $-1$ for failure, 0 for success.

The function {\tt wn\_set\_pixel\_colors} is used to set the colors
corresponding to a set of pixel values which have been allocated using
{\tt wn\_get\_pixels}.
For each of the {\tt ncolors} structures, {\tt wn\_set\_pixel\_colors} uses the color
levels {\tt co\_red}, {\tt co\_green} and {\tt co\_blue} to set the color
corresponding to the pixel value {\tt co\_pixel}.
The special pixel values {\tt WN\_FG} and {\tt WN\_BG} can be used with this
call to change the colors of the foreground and background pixels.

\sloppy{{\tt Wn\_get\_pixels\_by\_color}
is equivalent to {\tt wn\_get\_pixels} followed by {\tt wn\_set\_pixel\_colors}.
It allocates {\tt ncolors} colormap entries and sets the colors of the entries.}

The following function can be used to determine the colors currently
being used for each of a set of pixel values:
\funclabel{wn_get_pixel_colors}
\begin{verbatim}
void wn_get_pixel_colors(colors, ncolors)
color_t *colors;
int ncolors;
\end{verbatim}
For each of the {\tt ncolors} elements of the {\tt colors} array,
{\tt wn\_get\_pixel\_colors} sets {\tt co\_red}, {\tt co\_green} and {\tt co\_blue} to
the current color of a pixel with value {\tt co\_pixel}.
This call can be used to find the current foreground and background pixel
colors.

The final function should be used to free colormap entries
which are no longer needed
\funclabel{wn_free_pixels}
\begin{verbatim}
void wn_free_pixels(colors, ncolors)
color_t *colors;
int ncolors;
\end{verbatim}
For each of the {\tt ncolors} elements of the {\tt colors} array,
{\tt wn\_free\_pixels} frees the colormap entry specified by {\tt co\_pixel}.
%%%%%%%%%%%%%%%%%%%%%%%%%%%%%%%%%%%%%%%%
\subsection {Advanced colormap manipulation}
The {\tt wn\_get\_pixels} function simply allocates a given number of
colormap entries or pixels.  The pixel values chosen are random
as far as the application is concerned.
{\tt Wn} provides a more sophisticated colormap allocation function for
applications which do logical operations on pixels:
\funclabel{wn_get_pixels_and_planes}
\begin{verbatim}
int wn_get_pixels_and_planes(npixels, nplanes, contig, pixels, p_planes)
int npixels, nplanes, contig;
int *pixels, *p_planes;
\end{verbatim}
This function is almost identical to the X {\tt XAllocColorCells} function.
It allocates {\tt npixels} base pixel values and {\tt nplanes} planes.
The pixel values allocated are stored in the {\tt pixels} array, which
should have at least {\tt npixels} elements.
For each of the {\tt nplanes} planes requested, one bit will be set
in {\tt *p\_planes}.
If {\tt contig} is non zero, all the bits set in {\tt *p\_planes} will be
contiguous, otherwise they might not be.

For each base pixel value in {\tt pixels}, $2^{nplanes}$ pixel values
can be formed by {\tt or}ing in bits from {\tt *p\_planes}.

[ More explanation here ? ]
%%%%%%%%%%%%%%%%%%%%%%%%%%%%%%%%%
\subsection{Use of color hints}
On some color Suns the display has color planes and
a monochrome plane.  Displaying in the monochrome plane is much faster
than displaying the the color plane on these machines.
By default {\em wn} will run in the monochrome plane, which means
that the application will see a monochrome machine.

There is a function to override this and make {\em wn} use the color
planes if color is available:
\funclabel{wn_use_color_hint}
\begin{verbatim}
void wn_use_color_hint(use_color)
int use_color;
\end{verbatim}
If {\tt use\_color} is non zero, {\em wn} will use the color planes if
there is a choice of color on monochrome.  If {\tt use\_color} is zero,
{\tt wn} will go for monochrome if there is a choice (this is the default).

There are two other ways that the choice can be affected.
The environment variables {\tt USE\_COLOR} or {\tt USE\_MONO} can be set
to force the use of color or monochrome respectively, or the flags
{\tt -color} or {\tt -mono} can be given at run time to have the same effect
(see section \ref{munge_args}).
Both of these override any hint given by {\tt wn\_use\_color\_hint}.

Another idiosyncrasy of Suns is that allocating extra colormap cells can
cause glitches on the display.  An application which knows how many
colors it will need can avoid this by specifying in advance how many
cells it will request:
\funclabel{wn_npixels_hint}
\begin{verbatim}
void wn_npixels_hint(npixels)
int npixels;
\end{verbatim}
This indicates to {\em wn} that the application will need {\tt npixels}
pixels.
This function must be called before {\tt wn\_munge\_args} or
{\tt wn\_open\_stdwin}.
Note that this is only a hint -- it does not actually allocate the pixels
or guarantee that a request for this many pixels will succeed.
%%%%%%%%%%%%%%%%%%%%%%%%%%%%%%%%%%%%%%%%
\section{Subwindows}
You can create subwindows of top level windows.
In contrast to top level windows these are very lightweight objects.
They are usually used to divide up the application's main window into
regions, such as menus, scroll bars, etc.
\label{subwindows}
The function:
\funclabel{wn_create_subwin}
\begin{verbatim}
int wn_create_subwin(parwn, x, y, width, height, type)
int parwn;
int x, y, width, height, type;
\end{verbatim}
creates a subwindow of window {\tt parwn}, which can be either a
top level window or another subwindow (the second case allows
a hierarchy of subwindows to be built).
The subwindow's origin is {\tt x},{\tt y} relative to the origin
of {\tt parwn}, and its width and height are {\tt width} and {\tt height}.
{\tt Wn} only supports subwindows that are completely contained within
their parent windows -- you will get unpredictable results if any part
of the subwindow extends outside {\tt parwn}.
The {\tt type} argument specifies the type of the subwindow -- it is
either {\tt WN\_INPUT\_OUTPUT} or {\tt WN\_OUTPUT\_ONLY}.  More below.

Subwindows are mainly used for coordinate translation.  If you have, say,
a thumb bar at position {\em x},{\em y} in a window it can get 
tedious adding {\em x} and {\em y} to all coordinates when doing
graphical output to the thumb bar.
To avoid this you can create a subwindow at {\em x},{\em y} in the main
window and use its window number for graphical output in the thumb bar.
Subwindows do not do clipping, so the application still has to ensure that
output is restricted to the area of the subwindow.

There are two types of subwindow -- output only and input output.
If you just want to use the subwindow for output coordinate translation
as just described then you should use an output only subwindow,
created by giving {\tt wn\_create\_subwin} a {\tt type} argument of
{\tt WN\_OUTPUT\_ONLY}.
Subwindows can also be used to identify where graphical input occurred.
This is described in section \ref{events_and_subwindows}.
These subwindows should be created with a {\tt type} argument of
{\tt WN\_INPUT\_OUTPUT}.
%%%%%%%%%%%%%%%%%%%%%%%%%%%%%%%%%%%%%%%%
\subsection{Operations on subwindows}
There are several functions for changing the position and size of
subwindows:
\funclabel{wn_set_win_size}
\begin{verbatim}
void wn_set_win_size(wn, width, height)
int wn;
int width, height;
\end{verbatim}
Sets the size of {\tt wn} (which must refer to a subwindow) to {\tt width}
and {\tt height}.
\funclabel{wn_adjust_win_size}
\begin{verbatim}
void wn_adjust_win_size(wn, x_delta, y_delta, width_delta, height_delta)
int wn;
int x_delta, y_delta, width_delta, height_delta;
\end{verbatim}
Adjusts the position and offsets of subwindow {\tt wn} by the deltas
given, which may be positive or negative.  Use a zero delta for any
element which you don't want to change.
If window {\tt wn} has subwindows, their position relative to {\tt wn}
does not change (i.e. the children and parent move together as a unit).

Sometimes the application needs to translate coordinates from one
subwindow to another.
Use {\tt wn\_trans\_coords} to do this:
\funclabel{wn_trans_coords}
\begin{verbatim}
int wn_trans_coords(oldwn, x, y, newwn, p_x, p_y)
int oldwn;
int x, y;
int newwn;
int *p_x, *p_y;
\end{verbatim}
This function takes a pair of coordinates {\tt x},{\tt y} in window {\tt oldwn},
and sets {\tt *p\_x} and {\tt *p\_y} to the corresponding coordinates
in window {\tt newwn}.  It returns true if the point lies within window
{\tt newwn}.  Either or both of {\tt *p\_x} and {\tt *p\_y} may be {\tt NULL},
in which case they are not set.
%%%%%%%%%%%%%%%%%%%%%%%%%%%%%%%%%%%%%%%%
\section{Window system command line arguments}
\label{munge_args}
Most window systems define set of command line arguments which can
be used with any application run under the window system to control
things like the size of application's window, the standard font etc.
The set of these arguments for Sunview applications is listed
in the {\em Sunview} manual page.

Under X these arguments are less clearly defined --- the ones recognised
by {\em wn} are listed below.
The exact form of the arguments differs slightly between X10 and X11
for some arguments.
The table below shows the arguments recognised by {\em wn} with the
X10 and X11 forms.
%
%  Macro that produces an X geometry string
%
\newcommand{\xgeom}{{\it w}{\tt x}{\it h}{\tt +}{\it x}{\tt +}{\it y}}
%
\begin{center}
\begin{tabular}{|c|c|p{3.5in}|} \hline
{\bf X10} & {\bf X11} & {\bf Meaning} \\ \hline \hline
=\xgeom & {\tt -geometry} \xgeom & The initial
window size and position \\ \hline
\#\xgeom & -- & The initial icon size and position \\ \hline
{\it host}:0 & {\tt -display} {\it host}:0.0 & the host and display
where the program is to be run. The default display can also
be specified using the {\tt DISPLAY} environment variable. \\ \hline
\multicolumn{2}{|c|}{{\tt -fn} {\it font}} & The default font which will be used
when {\tt NULL} is used as a font pointer. {\it Font} should be
the name of a font which is available on the X server being used. \\ \hline
\multicolumn{2}{|c|}{{\tt -fg} {\it color}} & The foreground color when running
on a color display. This sets the color associated with the pixel value {\tt WN\_FG}
and can be either a named color (e.g. red) or an rgb color
value. \\ \hline
\multicolumn{2}{|c|}{{\tt -bg} {\it color}} &  The background color associated with the
pixel value {\tt WN\_BG}. \\ \hline
\end{tabular}\\
\end{center}
In order to make use of the command line entries the
application must call the function:
\funclabel{wn_munge_args}
\begin{verbatim}
int wn_munge_args(argc,argv)
int argc;
char **argv;
\end{verbatim}
{\tt Wn\_munge\_args} interprets any standard arguments and removes them from
the {\tt argv} array. The new value of {\tt argc} is returned and so the
arguments are given to {\em wn} by inserting the line
\begin{verbatim}
        argc = wn_munge_args(argc,argv);
\end{verbatim}
This must be positioned {\em before} the call to {\tt wn\_open\_stdwin}.

Most X applications also read configuration information from a defaults
database.
This database is stored in a file under X10, and in the server under X11.
See the X documentation for details on how to set up these databases.
{\em Wn} looks in the defaults database for the strings {\tt "Font"},
{\tt "Geometry"}, {\tt "Foreground"} and {\tt "Background"}.
These set the default font, window geometry, foreground color and
background color respectively.
Things specified via the argument list override things found in the
defaults database.

A program which uses its own parameters from the defaults database can
access them using the function:
\funclabel{wn_get_default}
\begin{verbatim}
char *wn_get_default(name)
char *name;
\end{verbatim}
where {\tt name} is the parameter name and the parameter value is returned.
If {\tt name} is not found in the defaults database {\tt wn\_get\_default}
returns {\tt NULL}.
For example suppose the defaults database contains the line
\begin{verbatim}
        foo.menuwidth:       34
\end{verbatim}
then a call of {\tt wn\_get\_default("menuwidth")} from within a program called
{\tt foo} will return the string {\tt "34"}. Note that {\tt wn\_munge\_args}
must be called before any attempt to obtain parameters from the defaults
database.

Under Sunview there is no defaults database and {\tt wn\_get\_default}
always returns {\tt NULL}.

Some applications invoke other graphical applications.
These child applications should be passed certain of the arguments
stripped out by {\tt wn\_munge\_args} (e.g. the display argument, if present).
To allow this {\em wn} saves arguments of this type that have been
removed by {\tt wn\_munge\_args}.  These saved arguments can be reinserted
into an array of arguments using {\tt wn\_unmunge\_args}:
\funclabel{wn_unmunge_args}
\begin{verbatim}
char **wn_unmunge_args(argv, pos)
char **argv;
int pos;
\end{verbatim}
This inserts the saved arguments {\tt argv} between {\tt argv[pos - 1]}
and {\tt argv[pos]}, and returns a new argument array.
The returned value should be passed to {\tt free} when it is finished with.
A typical code fragment for invoking a graphical child application
would be:
\begin{verbatim}
        < set up the child argument vector in child_argv >
        new_argv = wn_unmunge_args(child_argv, 1);
        < fork and exec child with new_argv >
        free((char *)new_argv);
\end{verbatim}
%%%%%%%%%%%%%%%%%%%%%%%%%%%%%%%%%%%%%%%%
\section{Fonts}
\label{fonts}
The {\em wn} font structure is defined in {\tt wn.h} as:
\begin{verbatim}
typedef struct {
        short ft_is_fixed_width;
        short ft_baseline;
        short ft_width;
        short ft_height;
        short ft_nchars;
        short *ft_width_tab
        mdfont_t ft_mdfont;
} font_t;
\end{verbatim}
The meanings of the fields are as follows:

\begin{center}
\begin{tabular}{|l|p{4in}|}
\hline
{\tt ft\_is\_fixed\_width} &
non zero if all the printable characters in the font have the same width. \\ \hline
{\tt ft\_baseline} &
the offset from the top of the font of the baseline. \\ \hline
{\tt ft\_width, ft\_height} &
the width and height of characters in the font.
For a variable width font, {\tt ft\_width} is the width of the widest
character in the font. \\ \hline
{\tt ft\_nchars} &
The number of characters in the font.
This will usually be at least 128, sometimes 256. \\ \hline
{\tt ft\_width\_tab} &
a table of individual widths of characters.
{\tt ft\_width\_tab{[i]}} is the width in pixels of the character with
ASCII code {\tt i}.
The table has {\tt ft\_nchars} elements, so the largest legal index is
{\tt ft\_nchars - 1}. \\ \hline
{\tt ft\_mdfont} &
a pointer to the machine dependent structure defining the font.
It is provided for applications that need to access local font structures
(e.g. library packages). \\ \hline
\end{tabular}\\[0.1in]
\nopagebreak
{\em The font structure fields}
\end{center}

A {\tt font\_t} pointer is returned by {\tt wn\_open\_font}, and can then be
used in calls to {\tt wn\_xputs} and other functions which use fonts.
The store occupied by a font can be released by calling {\tt wn\_close\_font}.

A {\tt NULL} {\tt font\_t} pointer refers to the system, or default, font.
It can be used in the same way as fonts obtained from {\tt wn\_open\_font}.
Thus programs using only the system font do not need to open fonts.
\label{wn_open_font}
\begin{verbatim}
font_t *wn_open_font(fontfile)
char *fontfile;
\end{verbatim}
{\tt Wn\_open\_font} reads the font in {\tt fontfile} and returns a pointer to a 
font structure.
The format of the font in {\tt fontfile} is machine dependent.
{\tt NULL} is returned if {\tt fontfile} cannot be read or does not contain a font
in an appropriate format.
\funclabel{wn_close_font}
\begin{verbatim}
void wn_close_font(font)
font_t *font;
\end{verbatim}
Closes (i.e. releases the store occupied by) the font referred
to by {\tt font}, which must have been obtained using {\tt wn\_open\_font}.
{\tt font} is no longer valid after this call.
\funclabel{wn_install_mdfont}
\begin{verbatim}
font_t *wn_install_mdfont(mdfont)
mdfont_t mdfont;
\end{verbatim}
Installs the font pointed to by {\tt mdfont} and returns a pointer to a
{\tt font\_t} structure for it.
{\tt mdfont} is assumed to point to a machine dependent structure
describing the font.
\funclabel{wn_get_sysfont}
\begin{verbatim}
font_t *wn_get_sysfont()
\end{verbatim}
Returns a font structure describing the system font.
Useful if you need information about the system font.
\funclabel{wn_set_sysfont}
\begin{verbatim}
void wn_set_sysfont(font)
font_t *font;
\end{verbatim}
Sets the system font to {\tt font}. 
Future calls to {\tt wn\_xputs} with a {\tt NULL} font structure pointer will
use this font.
If {\tt font} is {\tt NULL}, the system font is reset to its value before
any calls to {\tt wn\_set\_sysfont}.

There are some utility functions for converting character indexes
in strings in a given font to pixel offsets and vice versa.
All these functions work with variable width fonts as well as
fixed width fonts.
\funclabel{wn_strnwidth}
\begin{verbatim}
int wn_strnwidth(s, nchars, font)
char *s;
int nchars;
font_t *font;
\end{verbatim}
This function returns the width in pixels occupied by 
the first {\tt nchars} characters of string {\tt s}
were it to be drawn in font {\tt font}.
A {\tt NUL} character terminates the string if one occurs in the
first {\tt nchars} characters.
If {\tt nchars} is $-1$, characters up to the first {\tt NUL} are used.
\funclabel{wn_strwidth}
\begin{verbatim}
int wn_strwidth(s, font)
char *s;
font_t *font;
\end{verbatim}
This is equivalent to {\tt wn\_strnwidth(s, -1, font)}.

\funclabel{wn_strpos}
\begin{verbatim}
int wn_strpos(s, x, font, halfshift)
char *s;
int x;
font_t *font;
int halfshift;
\end{verbatim}
This function converts a pixel offset {\tt x} from the start of a string {\tt s}
rendered in font {\tt font} to an offset in bytes from the start of {\tt s}.
It can be used to determine which character of a string the user is pointing at.
If {\tt halfshift} is zero, a pixel is considered to be over a character if
it lies within the character.  If {\tt halfshift} is non zero, a pixel lies
over a character if is to the left of the horizontal midline of the character
and to the right of the horizontal midline of the previous character.
%%%%%%%%%%%%%%%%%%%%%%%%%%%%%%%%%%%%%%%%
\section{Cut and paste}
Most window systems support cutting and pasting of text between applications.
This means selecting a chunk of text within a window and clicking on a
mouse button to insert it as if it had been typed in the same or another
window.

At the application level, the window system usually maintains a global
object known as the {\em selection}, which contains the text that the
user last cut.  Some window systems have the concept of a selection type,
which allows more than just text to be cut and pasted.
{\em Wn} only has support for treating the selection as an array of bytes:
\funclabel{wn_get_selection}
\begin{verbatim}
void wn_get_selection(p_buf, p_nbytes)
char **p_buf;
int *p_nbytes;
\end{verbatim}

\funclabel{wn_set_selection}
\begin{verbatim}
void wn_set_selection(buf, nbytes)
char *buf;
int nbytes;
\end{verbatim}
{\tt Wn\_get\_selection} sets {\tt *p\_buf} to point at an array of bytes
containing the current selection, and sets {\tt *p\_nbytes} to the
number of bytes in the array.
The selection may contain embedded NUL characters.
The application should not modify the contents of the array.
If the user has not yet made any selection,
or the window manager does not support cut and paste, a zero size
selection will be returned.  

{\tt wn\_set\_selection} sets the current selection to the
first {\tt nbytes} bytes of array {\tt buf}.
If the window manager does not support cut and paste this function has
no effect.
%%%%%%%%%%%%%%%%%%%%%%%%%%%%%%%%%%%%%%%%
\section{Offscreen images}
\label{image}
Some applications need to manipulate off screen images, for example to
display an image loaded from a file, or to save a screen image to a
file.
The problem with doing this portably is that while most machines
support rasterops between the display and off screen memory,
different machines have different formats for the layout of an image
in memory.

We will use the term {\em bitmap} to denote any saved screen image,
whether binary or multi-plane.
{\em Wn} does not define a standard bitmap format; rather it supports
many formats and provides a function to convert between formats.
It is possible to ask for a bitmap to be converted to a format that is
efficient for the machine you are currently running on.
An application can create a bitmap in a format convenient for it, then
convert the bitmap to an efficient format for display.

A bitmap consists of a structure which describes the size and format
of the bitmap (the {\it bitmap header}) and a block of memory
containing the image (the {\it bitmap data}).
A field in the header points to the data.
The bitmap header is described by the following structure:
\begin{verbatim}
typedef struct {
        unsigned short *bm_data;
        char bm_bit_order;
        char bm_byte_order;
        char bm_pixel_format;
        unsigned char bm_flags;
        unsigned char bm_pflags;
        short bm_width;
        short bm_height;
        short bm_nplanes;
        short bm_xhot;
        short bm_yhot;
        short bm_lineinc;
        cmap_t *bm_colormap;
        long bm_cache_data;
} bitmap_t;
\end{verbatim}
The meanings of the fields are as follows:
\begin{tabbing}
\hspace{1.5in}\=\hspace{1.25in}\= \kill

{\tt bm\_data} \>
\parbox[t]{4.75in}{a pointer to the bitmap data
(an array of 16 bit unsigned integers defining the image)} \\[0.1in]

{\tt bm\_byte\_order} \>
\parbox[t]{4.75in}{the byte order for the image data.
The image data consists of 16 bit shorts, and there are two possible
ways of ordering the bytes in each short --- big endian or little
endian.
The possible values are:} \\[0.1in]

\> {\tt BM\_MSB\_FIRST} \>
\parbox[t]{3.5in}{Big endian order.
If the short is treated as an array of two bytes, the most significant byte
appears first.} \\[0.1in]

\>{\tt BM\_LSB\_FIRST} \>
\parbox[t]{3.5in}{Little endian order.
The least significant byte of the short appears first.} \\[0.1in]

\>{\tt BM\_MACHINE\_BYTE\_ORDER} \\ \>\>
\parbox[t]{3.5in}{The byte order of the shorts is that of the machine.
Use this for static bitmaps constructed from arrays of short constants in
the source.} \\[0.1in]

{\tt bm\_bit\_order} \>
\parbox[t]{4.75in}{the bit order for the image data.
The image data consists of 16 bit shorts, and there are two possible
ways of displaying the bits in each short --- big endian or little
endian.
The possible values are:} \\[0.1in]

\> {\tt BM\_BIT0\_RIGHT} \>
\parbox[t]{3.5in}{Big endian order. As the image is scanned from left to right,
the first bit displayed from a word is
the most significant bit (bit 15),
and the last is the least significant (bit 0).
Thus bit 0 of each word appears as the rightmost displayed.} \\[0.1in]

\>{\tt BM\_BIT0\_LEFT} \>
\parbox[t]{3.5in}{Little endian order.
The first bit displayed from a word, reading from left to right,
is the least significant bit.} \\[0.1in]

{\tt bm\_pixel\_format} \>
\parbox[t]{4.5in}{the pixel format of the bitmap.
The possible values are:} \\[0.1in]

\> {\tt BM\_XY\_PIXELS} \>
\parbox[t]{3.5in}{The bitmap data is organised as a set of planes (with the number of
planes being the same as the number of bits per pixel).
The most significant bits of all the pixels appear in the first plane,
the next most significant bit in the following plane, and so on until
the least significant bit.
Within each plane the pixels are packed 16 to a word. }\\[0.1in]

\>{\tt BM\_BYTE\_PIXELS} \>
\parbox[t]{3.5in}{This format is only supported for eight bit pixels.
Successive pixels are stored consecutively, with a single pixel in
each byte.
The bitmap data should be accessed in this case as an array of unsigned char.}\\[0.1in]

{\tt bm\_flags} \>
\parbox[t]{4.75in}{a set of flag bits --- see the description below. }\\[0.1in]

{\tt bm\_pflags} \>
\parbox[t]{4.75in}{a set of flag bits used internally by {\em wn}.
Don't change these.  In static bitmaps, or bitmaps created by
hand this field should be set to zero.}\\[0.1in]

{\tt bm\_width,bm\_height} \>
\parbox[t]{4.75in}{the width and height of the image in pixels. }\\[0.1in]

{\tt bm\_nplanes} \>
\parbox[t]{4.75in}{the number of bits per pixel for the bitmap. }\\[0.1in]

{\tt bm\_xhot, bm\_yhot} \>
\parbox[t]{4.75in}{the hotspot, if any, of the bitmap.
These are useful if the bitmap represents a cursor. }\\[0.1in]

{\tt bm\_lineinc} \>
\parbox[t]{4.75in}{the number of bytes between pixels
on adjacent lines and with the same x coordinate.
For example, if a bitmap in {\tt BM\_XY\_PIXELS} format has each line
of pixels padded to a whole number of
shorts, {\tt bm\_lineinc} will be {\tt ((bm\_width + 15) / 16) * 2}.
A {\tt BM\_BYTE\_PIXEL} bitmap will usually have {\tt bm\_lineinc} equal
to {\tt bm\_width}, although some machines require the line increment
for these bitmaps to be even. }\\[0.1in]

{\tt bm\_colormap} \>
\parbox[t]{4.75in}{This field is not used by {\em wn}.
It can be used by applications to store a colormap for a multiplane
bitmap.}\\[0.1in]

{\tt bm\_cache\_data} \>
\parbox[t]{4.75in}{used internally by {\em wn}.
It should not be examined or changed by the application.
In static bitmaps, or bitmaps created by hand,
this field should be set to zero. }\\[0.1in]
\end{tabbing}
The {\tt bm\_flags} field in a bitmap header contains a number of flag
bits which the application can set to modify the treatment of a bitmap.
The flags and their meanings are:
\begin{center}
\begin{tabular}{|l|p{4in}|} \hline

{\tt BM\_CAN\_FREE\_DATA} &
The bitmap data for this bitmap was obtained by calling {\tt malloc(3)}.
This flag is set by {\tt wn\_new\_bitmap}, and should be set by the
application if it constructs a bitmap with data obtained via {\tt malloc}.
The flag is checked by {\tt wn\_free\_bitmap} (see below).\\ \hline

{\tt BM\_CAN\_FREE\_HEADER} &
Similar to {\tt BM\_CAN\_FREE\_DATA}, but referring to the bitmap header.
This flag is set by {\tt wn\_new\_bitmap} and
{\tt wn\_data\_to\_bitmap}, and checked by {\tt wn\_free\_bitmap}.\\ \hline

{\tt BM\_CHOOSE\_FORMAT} &
{\em Wn} routines which do rasterops on bitmaps will convert a bitmap
to a format suitable for the machine before doing the operation,
and convert the format back afterwards.
If this flag is set, the bitmap format is not converted back afterwards,
so the expense of converting the format is only incurred on the first call.
Subsequent rasterop routines will find the bitmap already in the
correct format.\\ \hline

{\tt BM\_HAS\_HOTSPOT} &
This flag is ignored by {\em wn} --- it can be used by applications
to indicate whether a bitmap has a hotspot.\\ \hline
\end{tabular}\\[0.1in]
{\em Bitmap flags}
\end{center}
%%%%%%%%%%%%%%%%%%%%%%%%%%%%%%%%%%%%%%%%
\subsection{Changing the format of a bitmap}
\funclabel{wn_change_bitmap_format}
\begin{verbatim}
int wn_change_bitmap_format(bm, bit_order, byte_order, pixel_format, lineinc)
bitmap_t *bm;
int bit_order, byte_order, pixel_format, lineinc;
\end{verbatim}
Sets the {\tt bm\_bit\_order}, {\tt bm\_byte\_order},
{\tt bm\_pixel\_format} and {\tt bm\_lineinc}
fields of bitmap {\tt bm} to the values given, and reformats the bitmap
data accordingly.
See the descriptions above of the meanings of these fields.

A {\tt bit\_order} of {\tt BM\_NATURAL\_BIT\_ORDER} specifies the `natural' bit
order for the machine --- i.e. the bit order the machine uses.
The {\tt bm\_bit\_order} field will be set to {\tt BM\_BIT0\_RIGHT} or
{\tt BM\_BIT0\_LEFT}, depending on the machine.
Similarly, {\tt BM\_NATURAL\_BYTE\_ORDER} specifies the normal
byte order, {\tt BM\_NATURAL\_PIXEL\_FORMAT} specifies the machine's preferred
pixel format, and {\tt BM\_NATURAL\_LINEINC} the natural line increment for
the width and pixel format of the bitmap.

This routine is called implicitly by the bitmap rasterop routines for
bitmaps which are not in a format suitable for the machine.
This is expensive for large bitmaps, so you should ensure that a bitmap is
in the right order for the machine before doing lots of rasterops on it.
An easy way to do this is to set the {\tt BM\_CHOOSE\_FORMAT} bit in
{\tt bm\_flags} --- see the description of this flag above.
Alternatively you can call {\tt wn\_change\_bitmap\_format} with
{\tt BM\_NATURAL\_BIT\_ORDER}, {\tt BM\_NATURAL\_BYTE\_ORDER},
{\tt BM\_NATURAL\_PIXEL\_FORMAT} and {\tt BM\_NATURAL\_LINEINC}.
%%%%%%%%%%%%%%%%%%%%%%%%%%%%%%%%%%%%%%%%
\subsection{Creation and destruction of bitmaps}
\funclabel{wn_make_bitmap}
\begin{verbatim}
bitmap_t *wn_make_bitmap(width, height, nplanes, bit_order, pixel_format)
int width, height, nplanes, format;
\end{verbatim}
Creates a new bitmap and returns a pointer to it.
The new bitmap is {\tt width} pixels wide, {\tt height} pixels high, and has
{\tt nplanes} planes.
The bit order and pixel format of the bitmap are set to the values given.
The bitmap data is initialised to zero. 
\funclabel{wn_make_bitmap_from_data}
\begin{verbatim}
bitmap_t *wn_make_bitmap_from_data(width, height, nplanes, data, bit_order,
                                                 pixel_format, lineinc)
int width, height, nplanes;
unsigned short *data;
int bit_order, pixel_format, lineinc;
\end{verbatim}
Creates a bitmap header and sets the {\tt bm\_data} field of the new header
to {\tt data}.
{\tt bit\_order}, {\tt pixel\_format} and {\tt lineinc} should correspond
to the format of the data.
\funclabel{wn_static_bm}
\begin{verbatim}
bitmap_t wn_static_bm(width, height, nplanes, xhot, yhot, bit_order, data)
int width, height, nplanes, xhot, yhot, bit_order;
unsigned short data[];
\end{verbatim}
This macro generates a declaration for a static bitmap.
{\tt data} must be an array of {\tt (width + 15) / 16 * height} shorts.
{\tt bit\_order} should be one of {\tt BM\_BIT0\_LEFT} or {\tt BM\_BIT0\_RIGHT}.

The macro generates a declaration of the form:
\begin{verbatim}
{ field0, field1, ... }
\end{verbatim}
where {\tt field0}, {\tt field1} etc are initialisers for the various
fields of a bitmap header ({\tt bitmap\_t}, as described above).

For example, to statically define a 16 pixel square cursor called
{\tt cross\_cursor} with a hotspot at 4, 4 use something like:
\begin{verbatim}
unsigned short cross_cursor_data[16] = { num0, num1, ..., num15 };

bitmap_t cross_cursor = wn_static_bm(16, 16, 1, 4, 4, BM_BIT0_LEFT,
                                                        cross_cursor_data);
\end{verbatim}
Then {\tt \&cross\_cursor} (note the address operator) is a bitmap pointer
and can be passed to any of the bitmap routines.
{\tt wn\_static\_bm} sets the {\tt BM\_CHOOSE\_FORMAT} bit in {\tt bm\_flags},
which means that the bitmap will be converted to the natural format for the
machine on first use.
\funclabel{wn_free_bitmap}
\begin{verbatim}
void wn_free_bitmap(bm)
bitmap_t *bm;
\end{verbatim}
Frees a bitmap.
The bitmap data is freed (i.e. passed to {\tt free(3)}) if the
{\tt BM\_CAN\_FREE\_DATA} bit is set.
Similarly, the bitmap header is freed if the {\tt BM\_CAN\_FREE\_HEADER} bit is set.
If neither bit is set (as is the case with static bitmaps), this routine
has no effect.
\funclabel{wn_get_image}
\begin{verbatim}
void wn_get_image(wn, src_x, src_y, width, height, dbm, dst_x, dst_y, ropfunc)
int wn, src_x, src_y, width, height;
bitmap_t *dbm;
int dst_x, dst_y, ropfunc;
\end{verbatim}
Copies the area of window {\tt wn} given by {\tt src\_x}, {\tt src\_y},
{\tt width}, {\tt height} to {\tt dst\_x}, {\tt dst\_y} in bitmap {\tt dbm},
using rasterop function {\tt ropfunc}.

The destination is clipped to fit inside the bitmap given - i.e. there
is no danger of corrupting memory if the destination bitmap is too small or
if {\tt (dst\_x, dst\_y)} lies outside {\tt dbm}.

The destination bitmap {\tt dbm} must have the same depth (number of bits
per pixel) as the display, otherwise a fatal error results.
You can use {\tt wn\_get\_nplanes} (section \ref{wn_get_nplanes}
to find out the number of planes of the display.
\funclabel{wn_put_image}
\begin{verbatim}
void wn_put_image(sbm, src_x, src_y, width, height, wn, dst_x, dst_y, ropfunc, fg, bg)
bitmap_t *sbm;
int src_x, src_y, width, height;
int wn;
int dst_x, dst_y, ropfunc, fg, bg;
\end{verbatim}
Copies the area of bitmap {\tt sbm} given by {\tt src\_x}, {\tt src\_y},
{\tt width}, {\tt height} to {\tt dst\_x}, {\tt dst\_y} in window {\tt wn},
using rasterop function {\tt ropfunc}.

The source bitmap {\tt sbm} must have either the same number of planes as
the display, or one plane.
In the latter case, pixels on the display are set to {\tt fg} for set (1)
bits in the bitmap, and {\tt bg} for clear (0) bits (modulo {\tt ropfunc}).
\funclabel{wn_copy_image}
\begin{verbatim}
void wn_copy_image(sbm, src_x, src_y, width, height, dbm, dst_x, dst_y, ropfunc)
bitmap_t *sbm;
src_x, src_y, width, height;
bitmap_t *dbm;
int dst_x, dst_y, ropfunc;
\end{verbatim}
Copies the area of bitmap {\tt sbm} given by {\tt src\_x}, {\tt src\_y},
{\tt width}, {\tt height} to {\tt dst\_x}, {\tt dst\_y} in bitmap {\tt dbm},
using rasterop function {\tt ropfunc}.
The source and destination are clipped to fit {\tt sbm} and {\tt dbm} as in
{\tt wn\_get\_image}.

The source and destination bitmaps must have the same number of planes, and
the same pixel format.

\funclabel{wn_tile_area}
\begin{verbatim}
void wn_tile_area(wn, x, y, width, height, bm, ropfunc)
int wn;
int x, y, width, height, ropfunc;
\end{verbatim}

This function tiles an area specified by {\tt x}, {\tt y}, {\tt width} and
{\tt height} in window {\tt wn} with the image in bitmap {\tt bm}.
{\tt Bm} must be a 16 bit by 16 bit single plane bitmap.

The {\tt wn\_get\_image} and {\tt wn\_put\_image} functions make the bits
of the image available to the application.
This is slow for large images under X and is overkill if the application
just wants to save and restore an
area of the window without looking at the bits (for example, when
temporarily displaying a popup menu or dialog box).
For this reason a pair of specialised functions are provided to save
and restore an area of the window:
\funclabel{wn_save_area}
\begin{verbatim}
long wn_save_area(wn, x, y, width, height)
int wn;
int x, y, width, height;
\end{verbatim}
\funclabel{wn_restore_area}
\begin{verbatim}
void wn_restore_area(area_id)
long area_id;
\end{verbatim}
{\tt wn\_save\_area} saves the area of window {\tt wn} specified by
{\tt x}, {\tt y}, {\tt width} {\tt height} and returns a handle on the
saved area.
The returned handle can later be passed to {\tt wn\_restore\_area} which
restores the area to its state at the time {\tt wn\_save\_area} was called.
{\tt Wn\_restore\_area} can be called more than once, but it always puts
the image back in the same place.

When you have finished with an area (usually immediately after calling
{\tt wn\_restore\_area}) you should pass its area id to {\tt wn\_free\_area},
which frees the resources used by the area:
\funclabel{wn_free_area}
\begin{verbatim}
void wn_free_area(area_id)
long area_id;
\end{verbatim}
%%%%%%%%%%%%%%%%%%%%%%%%%%%%%%%%%%%%%%%%%%%%%%%%%%%%%
\section{Cursors}
These are the images that track the mouse.
%%%%%%%%%%%%%%%%%%%%%%%%%%%%%%%%%%%%%%%%
\subsection{Creating a cursor}
\label{setcurs}
The following function creates a cursor from a pair of bitmaps:
\funclabel{wn_create_cursor}
\begin{verbatim}
cursor_t wn_create_cursor(cursor_bitmap, mask_bitmap)
bitmap_t *cursor_bitmap, *mask_bitmap;
\end{verbatim}
This function does not cause the cursor to be displayed; it just
creates the cursor and returns a {\em cursor identifier} which
can be used to display the cursor.

{\tt Cursor\_bitmap} contains the image of the cursor.
{\tt Mask\_bitmap} acts as a stencil through which the image in
{\tt cursor\_bitmap} is painted.  If {\tt mask\_bitmap} is {\tt NULL}
then all bits of {\tt cursor\_bitmap} show.

Some window systems impose restrictions on the size of cursor that can be used.
A 16 bit by 16 bit cursor currently works on all window systems that {\em wn}
runs over.
A bitmap s clipped if it is too large to be used as a cursor.

\funclabel{wn_define_cursor}
\begin{verbatim}
void wn_define_cursor(wn, cursor_id)
int wn;
cursor_t cursor_id;
\end{verbatim}
Causes cursor {\tt cursor\_id} to be displayed in window {\tt wn}
whenever the mouse is in window {\tt wn}.
{\tt Cursor\_id} should be a cursor identifier returned by
{\tt wn\_create\_cursor}.

There is only one cursor for each top level window -- if {\tt wn}
is a subwindow (see section \ref{subwindows}) then the cursor is
set for the whole of the top level window that contains that
subwindow.
If you want subwindows to have individual cursors you will have to
change the cursor yourself when the mouse crosses a subwindow
boundary.

\funclabel{wn_get_window_cursor}
\begin{verbatim}
cursor_t wn_get_window_cursor(wn)
int wn;
\end{verbatim}
This function returns the cursor identifier of the cursor that is
currently defined for window {\tt wn}.
As with {\tt wn\_define\_cursor}, if {\tt wn} is a subwindow, the
cursor for its top level window is returned.
If no cursor has been defined for the window, {\tt wn\_get\_window\_cursor}
returns zero.

\funclabel{wn_free_cursor}
\begin{verbatim}
void wn_free_cursor(cursor)
cursor_t cursor;
\end{verbatim}
This function frees cursor {\tt cursor}.  {\tt Cursor} should not be used
after this call.  The effect of freeing a cursor that is currently in
use in a window is undefined.

Normally the application will create all of its cursors from bitmaps on
startup with {\tt wn\_create\_cursor}, then use {\tt wn\_define\_cursor} to
change cursors.
%%%%%%%%%%%%%%%%%%%%%%%%%%%%%%%%%%%%%%%%
\subsection{Special cursors}
\label{spcurs}
{\em Wn} provides special cursors --- rubber lines and boxes, and crosshairs.
\funclabel{wn_spcu}
\begin{verbatim}
void wn_spcu(wn, type_and_flags, x1, y1, x2, y2)
int wn;
unsigned type_and_flags;
int x1, y2, x2, y2;
\end{verbatim}
Sets or turns off a special cursor for window wn.
The normal cursor for the window is unaffected and appears as well
as the special cursor.
The {\tt type\_and\_flags} argument is made up of the bitwise OR of a
cursor type and zero or more flags.

The type part of {\tt type\_and\_flags} is one of:
\begin{list}{}{}
\item[\tt WN\_SC\_OFF]
Turn off any special cursor.
\item[\tt WN\_SC\_RECT]
A rectangle with one corner at {\tt x1, y1} and the diagonally opposite 
one at {\tt x2, y2}.
\item[\tt WN\_SC\_LINE]
A line from {\tt x1, y1} to {\tt x2, y2}.
\item[\tt WN\_SC\_CROSS]
Crosshairs centred on {\tt x1, y1}, made up of a horizontal and a vertical
line.
The {\tt x2, y2} arguments give the length and width of the lines.
\end{list}
The following flags may be ORed into the type argument:
\begin{list}{}{}
\item[\tt WN\_SC\_X1REL]
Make {\tt x1} relative to the cursor position (i.e. where the mouse is pointing).
\item[\tt WN\_SC\_Y1REL]
Make {\tt y1} cursor relative.
\item[\tt WN\_SC\_X2REL]
Make {\tt x2} cursor relative.
\item[\tt WN\_SC\_Y2REL]
Make {\tt y1} cursor relative.
\end{list}
As an example, to get a rubber line with one end fixed at 100, 150 and the
other tracking the cursor, use
\begin{verbatim}
wn_spcu(wn, WN_SC_LINE | WN_SC_X2REL | WN_SC_Y2REL, 100, 150, 0, 0);
\end{verbatim}
There are also some common combinations of type and flags defined:
\begin{list}{}{}
\item[\tt WN\_SC\_P1REL]
Equivalent to {\tt WN\_SC\_X1REL | WN\_SC\_Y1REL}.
\item[\tt WN\_SC\_P2REL]
Equivalent to {\tt WN\_SC\_X2REL | WN\_SC\_Y2REL}.
\item[\tt WN\_SC\_FRAME]
Wire frame --- {\tt x1, y1} and {\tt x2, y2} relative to the cursor.
\item[\tt WN\_SC\_RBOX]
Rubber box --- {\tt x1, y1} relative to the cursor, {\tt x2, y2} fixed.
\item[\tt WN\_SC\_RLINE]
Rubber line --- {\tt x1, y1} relative to the cursor, {\tt x2, y2} fixed.
\item[\tt WN\_SC\_XHAIR]
Crosshair on {\tt x1, y1}, relative to the cursor.
\end{list}
There are a number of macros defined for the more common special cursors.
They all call {\tt wn\_spcu}.
\funclabel{wn_sc_rbox}
\begin{verbatim}
void wn_sc_rbox(wn, x, y)
int wn, x, y;
\end{verbatim}
Sets a rubber box special cursor with one corner at {\tt x},{\tt y} and the
other tracking the cursor.
\funclabel{wn_sc_rline}
\begin{verbatim}
void wn_sc_rline(wn, x, y)
int wn, x, y;
\end{verbatim}
Sets a rubber line special cursor in window {\tt wn} with one end tracking
the cursor and the other fixed at {\tt x},{\tt y}.
\funclabel{wn_sc_hline}
\begin{verbatim}
void wn_sc_hline(wn, x)
int wn, x;
\end{verbatim}
Sets a horizontal rubber line special cursor for window {\tt wn}.
The {\em x} coordinate of one end of the line is fixed at {\tt x}.
The {\em x} coordinate of the other end tracks the cursor.
The {\em y} coordinate of both ends tracks the {\em y} coordinate of the
cursor.
\funclabel{wn_sc_vline}
\begin{verbatim}
void wn_sc_vline(wn, y)
int wn, y;
\end{verbatim}
As {\tt wn\_sc\_hline}, but the line is vertical, with the {\em y} coordinate of
both ends at {\tt y}.
\funclabel{wn_sc_off}
\begin{verbatim}
void wn_sc_off(wn)
int wn;
\end{verbatim}
Turns off any special cursor.
\begin{thebibliography}{99}
\bibitem{med3doc}  J D Bovey,
{\em A Tool for Providing Programs with Menus},
UKC Computing Laboratory internal document.

\bibitem{sunviewprog}
{\em The SunView Programmer's Guide},
Sun Microsystems Inc.

\bibitem{sunviewsysprog}
{\em The Sunview System Programmer's Guide},
Sun Microsystems Inc.

\bibitem{pixrect}
{\em The Pixrect Reference Manual},
Sun Microsystems Inc.

\bibitem{xlib} Robert Scheifler et. al.
{\em Xlib(1) -- C language interface to the X11 protocol},
Massachusetts Institute of Technology.

\end{thebibliography}
\newpage
\section{Index of functions}
\begin{small}
\begin{tabbing}
\hspace{5.25in}\={\bf \normalsize Page}\\
% @(#)wn-list.tex	1.1 4/11/91 (UKC)
% This document was written by John Bovey, University of Kent
%
\parbox[t]{5in}{\tt void wn\_adjust\_win\_size(wn, x\_delta, y\_delta, width\_delta, height\_delta)\\int wn;\\int x\_delta, y\_delta, width\_delta, height\_delta;}\> \pageref{wn_adjust_win_size}\\[0.1in]
\parbox[t]{5in}{\tt void wn\_await\_window\_size\_change(wn)\\int wn;}\> \pageref{wn_await_window_size_change}\\[0.1in]
\parbox[t]{5in}{\tt void wn\_bell(wn)\\int wn;}\> \pageref{wn_bell}\\[0.1in]
\parbox[t]{5in}{\tt void wn\_bold(wn, x, y, width, height)\\int wn;\\int x, y, width, height;}\> \pageref{wn_bold}\\[0.1in]
\parbox[t]{5in}{\tt void wn\_box\_round(wn, x, y, width, height, value)\\int wn;\\int x, y, width, height, value;}\> \pageref{wn_box_round}\\[0.1in]
\parbox[t]{5in}{\tt void wn\_bputs(wn, s, x, y)\\int wn;\\char *s;\\int x, y;}\> \pageref{wn_bputs}\\[0.1in]
\parbox[t]{5in}{\tt void wn\_btext(wn, s, x, y, fg\_value, bg\_value)\\int wn;\\char *s;\\int x, y;\\int fg\_value, bg\_value;}\> \pageref{wn_btext}\\[0.1in]
\parbox[t]{5in}{\tt int wn\_change\_bitmap\_format(bm, bit\_order, byte\_order, pixel\_format, lineinc)\\bitmap\_t *bm;\\int bit\_order, byte\_order, pixel\_format, lineinc;}\> \pageref{wn_change_bitmap_format}\\[0.1in]
\parbox[t]{5in}{\tt void wn\_close\_font(font)\\font\_t *font;}\> \pageref{wn_close_font}\\[0.1in]
\parbox[t]{5in}{\tt void wn\_close\_window(wn)\\int wn;}\> \pageref{wn_close_window}\\[0.1in]
\parbox[t]{5in}{\tt void wn\_copy\_image(sbm, src\_x, src\_y, width, height, dbm, dst\_x, dst\_y, ropfunc)\\bitmap\_t *sbm;\\src\_x, src\_y, width, height;\\bitmap\_t *dbm;\\int dst\_x, dst\_y, ropfunc;}\> \pageref{wn_copy_image}\\[0.1in]
\parbox[t]{5in}{\tt cursor\_t wn\_create\_cursor(cursor\_bitmap, mask\_bitmap)\\bitmap\_t *cursor\_bitmap, *mask\_bitmap;}\> \pageref{wn_create_cursor}\\[0.1in]
\parbox[t]{5in}{\tt int wn\_create\_subwin(parwn, x, y, width, height, type)\\int parwn;\\int x, y, width, height, type;}\> \pageref{wn_create_subwin}\\[0.1in]
\parbox[t]{5in}{\tt int wn\_create\_window()}\> \pageref{wn_create_window}\\[0.1in]
\parbox[t]{5in}{\tt void wn\_define\_cursor(wn, cursor\_id)\\int wn;\\cursor\_t cursor\_id;}\> \pageref{wn_define_cursor}\\[0.1in]
\parbox[t]{5in}{\tt void wn\_draw\_line(wn, x1, y1, x2, y2, value)\\int wn;\\int x1, y1, x2, y2, value;}\> \pageref{wn_draw_line}\\[0.1in]
\parbox[t]{5in}{\tt void wn\_free\_area(area\_id)\\long area\_id;}\> \pageref{wn_free_area}\\[0.1in]
\parbox[t]{5in}{\tt void wn\_free\_bitmap(bm)\\bitmap\_t *bm;}\> \pageref{wn_free_bitmap}\\[0.1in]
\parbox[t]{5in}{\tt void wn\_free\_cursor(cursor)\\cursor\_t cursor;}\> \pageref{wn_free_cursor}\\[0.1in]
\parbox[t]{5in}{\tt void wn\_free\_pixels(colors, ncolors)\\color\_t *colors;\\int ncolors;}\> \pageref{wn_free_pixels}\\[0.1in]
\parbox[t]{5in}{\tt char *wn\_get\_default(name)\\char *name;}\> \pageref{wn_get_default}\\[0.1in]
\parbox[t]{5in}{\tt int wn\_get\_fd\_mask()}\> \pageref{wn_get_fd_mask}\\[0.1in]
\parbox[t]{5in}{\tt int wn\_get\_height(wn)\\int wn;}\> \pageref{wn_get_height}\\[0.1in]
\parbox[t]{5in}{\tt void wn\_get\_image(wn, src\_x, src\_y, width, height, dbm, dst\_x, dst\_y, ropfunc)\\int wn, src\_x, src\_y, width, height;\\bitmap\_t *dbm;\\int dst\_x, dst\_y, ropfunc;}\> \pageref{wn_get_image}\\[0.1in]
\parbox[t]{5in}{\tt int wn\_get\_nplanes()}\> \pageref{wn_get_nplanes}\\[0.1in]
\parbox[t]{5in}{\tt void wn\_get\_pixel\_colors(colors, ncolors)\\color\_t *colors;\\int ncolors;}\> \pageref{wn_get_pixel_colors}\\[0.1in]
\parbox[t]{5in}{\tt int wn\_get\_pixels(colors, ncolors)\\color\_t *colors;\\int ncolors;}\> \pageref{wn_get_pixels}\\[0.1in]
\parbox[t]{5in}{\tt int wn\_get\_pixels\_and\_planes(npixels, nplanes, contig, pixels, p\_planes)\\int npixels, nplanes, contig;\\int *pixels, *p\_planes;}\> \pageref{wn_get_pixels_and_planes}\\[0.1in]
\parbox[t]{5in}{\tt int wn\_get\_pixels\_by\_color(colors, ncolors)\\color\_t *colors;\\int ncolors;}\> \pageref{wn_get_pixels_by_color}\\[0.1in]
\parbox[t]{5in}{\tt int wn\_get\_resize\_event(wn)\\int wn;}\> \pageref{wn_get_resize_event}\\[0.1in]
\parbox[t]{5in}{\tt void wn\_get\_selection(p\_buf, p\_nbytes)\\char **p\_buf;\\int *p\_nbytes;}\> \pageref{wn_get_selection}\\[0.1in]
\parbox[t]{5in}{\tt font\_t *wn\_get\_sysfont()}\> \pageref{wn_get_sysfont}\\[0.1in]
\parbox[t]{5in}{\tt int wn\_get\_width(wn)\\int wn;}\> \pageref{wn_get_width}\\[0.1in]
\parbox[t]{5in}{\tt long wn\_get\_win\_data(wn)\\int wn;}\> \pageref{wn_get_win_data}\\[0.1in]
\parbox[t]{5in}{\tt cursor\_t wn\_get\_window\_cursor(wn)\\int wn;}\> \pageref{wn_get_window_cursor}\\[0.1in]
\parbox[t]{5in}{\tt void wn\_get\_window\_size(wn, p\_width, p\_height)\\int wn;\\int *p\_width, *p\_height;}\> \pageref{wn_get_window_size}\\[0.1in]
\parbox[t]{5in}{\tt int wn\_get\_wm\_fds()}\> \pageref{wn_get_wm_fds}\\[0.1in]
\parbox[t]{5in}{\tt int wn\_get\_wm\_type()}\> \pageref{wn_get_wm_type}\\[0.1in]
\parbox[t]{5in}{\tt void wn\_inmode(wn, mode)\\int wn, mode;}\> \pageref{wn_inmode}\\[0.1in]
\parbox[t]{5in}{\tt font\_t *wn\_install\_mdfont(mdfont)\\mdfont\_t mdfont;}\> \pageref{wn_install_mdfont}\\[0.1in]
\parbox[t]{5in}{\tt void wn\_invert\_area(wn, x, y, width, height)\\int wn;\\int x, y, width, height;}\> \pageref{wn_invert_area}\\[0.1in]
\parbox[t]{5in}{\tt void wn\_invert\_box(wn, x, y, width, height)\\int wn;\\int x, y, width, height;}\> \pageref{wn_invert_box}\\[0.1in]
\parbox[t]{5in}{\tt void wn\_invert\_line(wn, x1, y1, x2, y2)\\int wn;\\int x1, x2, y1, y2;}\> \pageref{wn_invert_line}\\[0.1in]
\parbox[t]{5in}{\tt int wn\_is\_open(wn)\\int wn;}\> \pageref{wn_is_open}\\[0.1in]
\parbox[t]{5in}{\tt int wn\_last\_rop\_was\_damaged(wn)\\int wn;}\> \pageref{wn_last_rop_was_damaged}\\[0.1in]
\parbox[t]{5in}{\tt bitmap\_t *wn\_make\_bitmap(width, height, nplanes, bit\_order, pixel\_format)\\int width, height, nplanes, format;}\> \pageref{wn_make_bitmap}\\[0.1in]
\parbox[t]{5in}{\tt bitmap\_t *wn\_make\_bitmap\_from\_data(width, height, nplanes, data, bit\_order,\\                                                 pixel\_format, lineinc)\\int width, height, nplanes;\\unsigned short *data;\\int bit\_order, pixel\_format, lineinc;}\> \pageref{wn_make_bitmap_from_data}\\[0.1in]
\parbox[t]{5in}{\tt void wn\_mono\_rop(wn, sx, sy, width, height, dx, dy, ropfunc)\\int wn;\\int sx, sy, width, height, dx, dy, ropfunc;}\> \pageref{wn_mono_rop}\\[0.1in]
\parbox[t]{5in}{\tt void wn\_mono\_text(wn, font, s, x, y, ropfunc, use\_which)\\int wn;\\font\_t font;\\char *s;\\int x, y, ropfunc, use\_which;}\> \pageref{wn_mono_text}\\[0.1in]
\parbox[t]{5in}{\tt void wn\_move\_area(wn, src\_x, src\_y, width, height, dst\_x, dst\_y, value)\\int wn;\\int src\_x, src\_y, width, height, dst\_x, dst\_y, value;}\> \pageref{wn_move_area}\\[0.1in]
\parbox[t]{5in}{\tt int wn\_munge\_args(argc,argv)\\int argc;\\char **argv;}\> \pageref{wn_munge_args}\\[0.1in]
\parbox[t]{5in}{\tt void wn\_next\_event(wn, mask, event)\\int wn, mask;\\event\_t *event;}\> \pageref{wn_next_event}\\[0.1in]
\parbox[t]{5in}{\tt void wn\_npixels\_hint(npixels)\\int npixels;}\> \pageref{wn_npixels_hint}\\[0.1in]
\parbox[t]{5in}{\tt int wn\_open\_stdwin()}\> \pageref{wn_open_stdwin}\\[0.1in]
\parbox[t]{5in}{\tt void wn\_pushback\_event(event)\\event\_t *event;}\> \pageref{wn_pushback_event}\\[0.1in]
\parbox[t]{5in}{\tt void wn\_put\_image(sbm, src\_x, src\_y, width, height, wn, dst\_x, dst\_y, ropfunc, fg, bg)\\bitmap\_t *sbm;\\int src\_x, src\_y, width, height;\\int wn;\\int dst\_x, dst\_y, ropfunc, fg, bg;}\> \pageref{wn_put_image}\\[0.1in]
\parbox[t]{5in}{\tt void wn\_restore\_area(area\_id)\\long area\_id;}\> \pageref{wn_restore_area}\\[0.1in]
\parbox[t]{5in}{\tt void wn\_rop(wn, src\_x, src\_y, width, height, dst\_x, dst\_y)\\int wn;\\int src\_x, src\_y, width, height, dst\_x, dst\_y;}\> \pageref{wn_rop}\\[0.1in]
\parbox[t]{5in}{\tt long wn\_save\_area(wn, x, y, width, height)\\int wn;\\int x, y, width, height;}\> \pageref{wn_save_area}\\[0.1in]
\parbox[t]{5in}{\tt void wn\_sc\_hline(wn, x)\\int wn, x;}\> \pageref{wn_sc_hline}\\[0.1in]
\parbox[t]{5in}{\tt void wn\_sc\_off(wn)\\int wn;}\> \pageref{wn_sc_off}\\[0.1in]
\parbox[t]{5in}{\tt void wn\_sc\_rbox(wn, x, y)\\int wn, x, y;}\> \pageref{wn_sc_rbox}\\[0.1in]
\parbox[t]{5in}{\tt void wn\_sc\_rline(wn, x, y)\\int wn, x, y;}\> \pageref{wn_sc_rline}\\[0.1in]
\parbox[t]{5in}{\tt void wn\_sc\_vline(wn, y)\\int wn, y;}\> \pageref{wn_sc_vline}\\[0.1in]
\parbox[t]{5in}{\tt int (*wn\_set\_abort\_func(wn, func))()\\int wn;\\int (*func)();}\> \pageref{wn_set_abort_func}\\[0.1in]
\parbox[t]{5in}{\tt void wn\_set\_area(wn, x, y, width, height, value)\\int wn;\\int x, y, width, height, value;}\> \pageref{wn_set_area}\\[0.1in]
\parbox[t]{5in}{\tt void wn\_set\_fd\_mask(mask)\\int mask;}\> \pageref{wn_set_fd_mask}\\[0.1in]
\parbox[t]{5in}{\tt void wn\_set\_pixel\_colors(colors, ncolors)\\color\_t *colors;\\int ncolors;}\> \pageref{wn_set_pixel_colors}\\[0.1in]
\parbox[t]{5in}{\tt void wn\_set\_selection(buf, nbytes)\\char *buf;\\int nbytes;}\> \pageref{wn_set_selection}\\[0.1in]
\parbox[t]{5in}{\tt void wn\_set\_sysfont(font)\\font\_t *font;}\> \pageref{wn_set_sysfont}\\[0.1in]
\parbox[t]{5in}{\tt long wn\_set\_win\_data(wn, data)\\int wn;\\long data;}\> \pageref{wn_set_win_data}\\[0.1in]
\parbox[t]{5in}{\tt void wn\_set\_win\_size(wn, width, height)\\int wn;\\int width, height;}\> \pageref{wn_set_win_size}\\[0.1in]
\parbox[t]{5in}{\tt void wn\_shade\_area(wn, x, y, width, height, shade, ropfunc)\\int wn;\\int x, y, width, height, ropfunc;\\short shade;}\> \pageref{wn_shade_area}\\[0.1in]
\parbox[t]{5in}{\tt void wn\_show\_updates(wn)\\int wn;}\> \pageref{wn_show_updates}\\[0.1in]
\parbox[t]{5in}{\tt void wn\_spcu(wn, type\_and\_flags, x1, y1, x2, y2)\\int wn;\\unsigned type\_and\_flags;\\int x1, y2, x2, y2;}\> \pageref{wn_spcu}\\[0.1in]
\parbox[t]{5in}{\tt bitmap\_t wn\_static\_bm(width, height, nplanes, xhot, yhot, bit\_order, data)\\int width, height, nplanes, xhot, yhot, bit\_order;\\unsigned short data[];}\> \pageref{wn_static_bm}\\[0.1in]
\parbox[t]{5in}{\tt int wn\_strnwidth(s, nchars, font)\\char *s;\\int nchars;\\font\_t *font;}\> \pageref{wn_strnwidth}\\[0.1in]
\parbox[t]{5in}{\tt int wn\_strpos(s, x, font, halfshift)\\char *s;\\int x;\\font\_t *font;\\int halfshift;}\> \pageref{wn_strpos}\\[0.1in]
\parbox[t]{5in}{\tt int wn\_strwidth(s, font)\\char *s;\\font\_t *font;}\> \pageref{wn_strwidth}\\[0.1in]
\parbox[t]{5in}{\tt void wn\_text(wn, font, s, x, y, fg\_value, bg\_value, use\_which)\\int wn;\\font\_t *font;\\char *s;\\int x, y, value, use\_which;}\> \pageref{wn_text}\\[0.1in]
\parbox[t]{5in}{\tt void wn\_tile\_area(wn, x, y, width, height, bm, ropfunc)\\int wn;\\int x, y, width, height, ropfunc;}\> \pageref{wn_tile_area}\\[0.1in]
\parbox[t]{5in}{\tt void wn\_tputs(wn, s, x, y)\\int wn;\\char *s;\\int x, y;}\> \pageref{wn_tputs}\\[0.1in]
\parbox[t]{5in}{\tt int wn\_trans\_coords(oldwn, x, y, newwn, p\_x, p\_y)\\int oldwn;\\int x, y;\\int newwn;\\int *p\_x, *p\_y;}\> \pageref{wn_trans_coords}\\[0.1in]
\parbox[t]{5in}{\tt void wn\_ttext(wn, s, x, y, fg\_value, bg\_value)\\int wn;\\char *s;\\int x, y;\\int fg\_value, bg\_value;}\> \pageref{wn_ttext}\\[0.1in]
\parbox[t]{5in}{\tt char **wn\_unmunge\_args(argv, pos)\\char **argv;\\int pos;}\> \pageref{wn_unmunge_args}\\[0.1in]
\parbox[t]{5in}{\tt wn\_updating\_on(wn)\\int wn;}\> \pageref{wn_updating_on}\\[0.1in]
\parbox[t]{5in}{\tt void wn\_use\_color\_hint(use\_color)\\int use\_color;}\> \pageref{wn_use_color_hint}\\[0.1in]
\parbox[t]{5in}{\tt void wn\_wait\_for\_release\_of(wn, button)\\int wn, button;}\> \pageref{wn_wait_for_release_of}\\[0.1in]
\parbox[t]{5in}{\tt void wn\_warp\_mouse(wn, x, y)\\int wn;\\int x, y;}\> \pageref{wn_warp_mouse}\\[0.1in]

\end{tabbing}
\end{small}
\end{document}
